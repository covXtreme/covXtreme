\usepackage{float}
\usepackage{graphicx}
\usepackage{subcaption}
% \usepackage{amsthm} % Needed to typeset theorem environments.
%\usepackage{MnSymbol}
\newcommand{\Dft}[2]{D_{#1,#2}}%d_N(\omega\mid X_\Delta)
\newcommand{\periodogram}[2]{I}%{I_{#1,#2}}
\newcommand{\sumXt}[2]{\sum^{#1}_{t=1}X_{t#2}\exp\{-it#2\omega\}}
\newcommand{\sumOmega}[3]{\sum_{\omega\in\Omega_#1}#3\exp\{it#2\omega\}}
\newcommand{\sumDo}[2]{\sumOmega{#1}{#2}{\Dft{#1}{#2}(\omega)}}
\newcommand{\mysqrt}[1]{\left(#1\right)^{\frac{1}{2}}}
\newcommand{\Exponential}[1]{\text{Exponential}\left(#1\right)}
\newcommand{\cNorm}[2]{\mathcal{CN}\left(#1,#2\right)}
\newcommand{\Norm}[2]{\mathcal{N}\left(#1,#2\right)}
\newcommand{\acf}{c}
%%
% Commands for special set of numbers.
\newcommand{\NN}{\mathbb{N}}
\newcommand{\ZZ}{\mathbb{Z}}
\newcommand{\RR}{\mathbb{R}}
\newcommand{\PP}[1]{\mathbb{P}\left( #1 \right)}
\newcommand{\EE}[1]{\mathbb{E} \left[ #1 \right]}
\newcommand{\CC}{\mathbb{C}}
\newcommand{\QQ}{\mathbb{Q}}
\newcommand{\e}[1]{{\textrm{e}}^{#1}}
% Commands for useful operators.
\newcommand{\di}{\text{diag}}
\newcommand{\de}{\mathrm{d}}
\newcommand{\si}{\Sigma}
\newcommand{\del}{\Delta}
\newcommand{\cov}[1]{\, {\rm cov}\left( #1 \right) }
\newcommand{\var}[1]{\, {\rm var}\left( #1 \right) }
\newcommand{\cum}[1]{\, {\rm cum}\left( #1 \right) }
\newcommand{\varhat}[1]{\, {\rm \hat{var}}\left( #1 \right) }
\newcommand{\covhat}[1]{\, {\rm \hat{cov}}\left( #1 \right) }
\newcommand{\toas}{\xrightarrow{\text{a.s.}}}
\newcommand{\todis}{\xrightarrow{\text{d}}}
\newcommand{\supp}{\text{supp}}
\newcommand{\abs}{\text{abs}}
\newcommand{\doubint}{\!\int\!\!\!\int}
\DeclareMathOperator*{\argmin}{argmin} % no space, limits underneath in displays
\DeclareMathOperator*{\argmax}{argmax} % no space, limits underneath in displays
% Command for text within a math sub/super-script.
\newcommand{\stext}[1]{\text{\scriptsize{#1}}}



% commands for model
\newcommand{\expbit}{\exp\left \{-\frac{r}{s}\left(\frac{\omega}{\omega_p}\right)^{-s}\right \}}
% \newcommand{\parbit}{\omega,\alpha,\omega_p,\gamma,r}
% \newcommand{\minusparbit}{-\omega,\alpha,\omega_p,\gamma,r}
\newcommand{\parbit}{\omega\mid\theta}
\newcommand{\minusparbit}{-\omega\mid\theta}
\newcommand{\dbit}{\delta(\parbit)}
\newcommand{\sigbit}{\sigma(\parbit)}
\newcommand{\fbit}{f_G(\parbit)}
\newcommand{\gbit}{g(\parbit)}
\newcommand{\Sbit}{S_G(\parbit)}
\newcommand{\pard}[1]{\frac{\partial}{\partial #1}}
\newcommand{\negderiv}[1]{\pard{#1}\fbit&=\pard{#1}f_G(\minusparbit)}

% commands that need notational checking
\newcommand{\XN}{\bX_{\Delta,N}}
\newcommand{\bX}{\boldsymbol{X}}
\newcommand{\bx}{\boldsymbol{x}}

% Theorem definitions/
% \newtheorem{lemma}{Lemma}[section]
% \newtheorem{theorem}[lemma]{Theorem}
% \newtheorem{proposition}[lemma]{Proposition}
% \newtheorem{corollary}[lemma]{Corollary}
\newtheorem{proposition}{Proposition}
\newproof{proof}{Proof}

%% matlab logo
\newcommand{\matlab}{MATLAB }