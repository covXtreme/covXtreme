%%%%%%%%%%%%%%%%%%%%%%%%%%%%%%%%%%%%%%%%%%%%%%%%%%%%%%%%%%%
\documentclass[a4paper,10pt]{article}
\usepackage{amsmath,amssymb,amsfonts}
\usepackage{natbib,upgreek,mathtools}
\usepackage{rotating, caption}
\usepackage{graphicx}
\usepackage{subcaption}
\usepackage{cprotect}
\usepackage{color}
\usepackage{fullpage}
\usepackage{titlesec}
\usepackage{hyperref}
\usepackage{bbm}
\usepackage{amsmath}
\usepackage[dvipsnames]{xcolor}
\newcommand{\sectionbreak}{\clearpage}
%\usepackage[nolists]{endfloat}
\setcounter{section}{-1}%%%%%%%%%%%%%%%%%%%%%%%%%%%%%%%%%%%%%%%%%%%%%%%%%%%%%%%%%%%
%Phil's LaTeX preamble file
%%%%%%%%%%%%%%%%%%%%%%%%%%%%%%%%%%%%%%%%%%%%%%%%%%%%%%%%%%%

%%%%%%%%%%%%%%%%%%%%%%%%%%%%%%%%%%%%%%%%%%%%%%%%%%%%%%%%%%%
%Itemise
\newcommand{\pbi}{\begin{itemize}}
\newcommand{\pei}{\end{itemize}}
\newcommand{\pii}{\item}
%Centre
\newcommand{\pbc}{\begin{center}}
\newcommand{\pec}{\end{center}}
%Equation array
\newcommand{\pbe}{\begin{eqnarray*}}
\newcommand{\pee}{\end{eqnarray*}}
%%%%%%%%%%%%%%%%%%%%%%%%%%%%%%%%%%%%%%%%%%%%%%%%%%%%%%%%%%%

%%%%%%%%%%%%%%%%%%%%%%%%%%%%%%%%%%%%%%%%%%%%%%%%%%%%%%%%%%%
%Functions
%Median over something
\newcommand\med[1]{\underset{#1}{\mathrm{med}}}
%Underline of something
\newcommand{\un}[1]{\boldsymbol{#1}}
%%%%%%%%%%%%%%%%%%%%%%%%%%%%%%%%%%%%%%%%%%%%%%%%%%%%%%%%%%%

%%%%%%%%%%%%%%%%%%%%%%%%%%%%%%%%%%%%%%%%%%%%%%%%%%%%%%%%%%%
%Shortcut notation
\let\hat=\widehat
\let\geq=\geqslant
\let\leq=\leqslant
\providecommand{\Pr}{\mathbb{Pr}} %Probability
\newcommand{\pms}{\quad}
%approximately equal for large n
\newcommand{\asseq}{\,\stackrel{{\rm \text{large } n}}{\approx}\,}
%equality by definition
\newcommand{\defeq}{\,\stackrel{{\rm \vartriangle}}{=}\,}
%independently distributed
\newcommand{\simindep}{\,\stackrel{{\rm indep}}{\sim}\,}
%argmin
\DeclareMathOperator*{\argmin}{arg\,\min}
%%%%%%%%%%%%%%%%%%%%%%%%%%%%%%%%%%%%%%%%%%%%%%%%%%%%%%%%%%%

%%%%%%%%%%%%%%%%%%%%%%%%%%%%%%%%%%%%%%%%%%%%%%%%%%%%%%%%%%%
%Dot
\providecommand{\Xd}{\dot{X}}
\providecommand{\Yd}{\dot{Y}}
\providecommand{\zd}{\dot{z}}
\providecommand{\yd}{\dot{y}}
\providecommand{\xd}{\dot{x}}
\providecommand{\thetad}{\dot{\theta}}
\providecommand{\phid}{\dot{\phi}}
\providecommand{\nd}{{\dot{n}}}
%Tilde
\providecommand{\rhot}{\tilde{\rho}}
\providecommand{\sigmat}{\tilde{\sigma}}
\providecommand{\xit}{\tilde{\xi}}
\providecommand{\ut}{{\tilde{u}}}
\providecommand{\qt}{{\tilde{q}}}
\providecommand{\Qt}{{\tilde{Q}}}
\providecommand{\qb}{{\breve{q}}}
\providecommand{\Qb}{{\breve{Q}}}
%%%%%%%%%%%%%%%%%%%%%%%%%%%%%%%%%%%%%%%%%%%%%%%%%%%%%%%%%%%

%%%%%%%%%%%%%%%%%%%%%%%%%%%%%%%%%%%%%%%%%%%%%%%%%%%%%%%%%%%
%Variable shortcuts for extremes work
\providecommand{\cvr}{{\theta,\phi}}
\providecommand{\cvrA}{{\theta \text{ and } \phi}}
\providecommand{\prm}{{\psi,\rho,\xi \text{ and } \sigma}}
%%%%%%%%%%%%%%%%%%%%%%%%%%%%%%%%%%%%%%%%%%%%%%%%%%%%%%%%%%%

%%%%%%%%%%%%%%%%%%%%%%%%%%%%%%%%%%%%%%%%%%%%%%%%%%%%%%%%%%%
%Text layout
%\onehalfspacing            %line spacing
\setlength{\parindent}{0cm} %extent of text at end of paragraph
\setlength{\parskip}{0em}   %length of gap following paragraph end (use \par to force new paragraph)
%%%%%%%%%%%%%%%%%%%%%%%%%%%%%%%%%%%%%%%%%%%%%%%%%%%%%%%%%%%

%%%%%%%%%%%%%%%%%%%%%%%%%%%%%%%%%%%%%%%%%%%%%%%%%%%%%%%%%%%
%Page size
%\oddsidemargin  -0.7in
%\evensidemargin -0.7in
%\textwidth      7.6in
%\headheight     0.0in
%\topmargin      -1.0in
%\textheight     10.5in
%%%%%%%%%%%%%%%%%%%%%%%%%%%%%%%%%%%%%%%%%%%%%%%%%%%%%%%%%%%


\begin{document}

\title{\textbf{Contour estimation using Penalised Piecewise Constant marginal and conditional extreme value models}}
\author{Emma Ross\thanks{Corresponding author. Email: {\tt e.ross@shell.com}}, David Randell, Philip Jonathan}
%\author{Bob}
\date{\today}
%\date{} %nodate
\maketitle
%-------------------------------------------------------------------------------

%-------------------------------------------------------------------------------
\begin{abstract}
	This report describes the Penalised Piecewise Constant (PPC) model and software for estimation of environmental design contours using the conditional extremes model of \cite{HffTwn04}. The sample is composed of peaks over threshold values for both a conditioning variate and its \emph{associated} conditioned variates. Each pair is allocated to a particular \emph{covariate bin}; all (joint) observations with the same covariate bin are assumed to have common extreme value characteristics. The non-stationary marginal extreme value characteristics of each variate is estimated using roughness-penalised maximum likelihood estimation using a generalised Pareto (GP) model above the threshold and gamma below. The extremal dependence structure between the variates on a transformed standard scale (Gumbel or Laplace) is then estimated using a conditional extremes model, also piecewise non-stationary with respect to covariates. Different approaches to contour estimation, generally reliant on simulation under the fitted models, are outlined.
	\newline \newline \newline \newline \newline \newline
	\noindent\textbf{Major Updates Since Previous Release}
	\begin{itemize}
		\item Extended to more than one covariate and non-periodic covariates;
		\item Hefferenan and Tawn model extended from bivariate to multivariate cases; 
		\item Marginal model now fits a Gamma distribution below the threshold, instead of an empirical distribution;
		\item Margins can now be transformed to either Laplace or Gumbel scale (the Laplace scale more naturally handles negative dependence);
		\item Improvements to memory usage and speed of return value estimation;
		\item Use of importance sampling to speed up computation of contours;
		\item Alteration to contouring options: empirical and Heffernan and Tawn (H\&T) density contouring methods merged; radial quantile method replaced with Huseby contour (similar concept but more rigorously defined); \textcolor{red}{}
		\item Contour levels are derived from the same return periods used in the Marginal and H\&T stages, i.e. no longer defined by probability level on the conditioned variable
	\end{itemize}
\end{abstract}

%
\tableofcontents

%\listoffigures




\section{Introduction}
%
The conditional extremes model of \cite{HffTwn04}, and extensions such as \cite{JntEwnRnd14}, \cite{KefPpsTan13} provide a framework to estimate multivariate extremal dependence in the presence of covariates, and hence to estimate design contours and other statistics of interest in metocean design. The approach is motivated by an asymptotic form for the limiting conditional distribution of one or more conditioned random variables given a large value of a conditioning variable. An outline of the approach is given by \cite{JntFlnEwn10}. Conditions for the asymptotic argument to hold have been explored by \cite{HffRsn07}.

Suppose we want to estimate design contours using the conditional extremes model for bivariate peaks over threshold of random variables $\dot{Y}_1$ and $\dot{Y}_2$. These variables might be significant wave height $H_S$ and associated peak period $T_P$, and the dependence between them might be non-stationary with respect to covariates $\boldsymbol{X}$ such as season or storm direction. We then want to simulate realisations under the model, and use the simulation to estimate design contours. We propagate uncertainties due to tuning parameter choice (specifically, threshold levels for marginal and dependence models) and sampling throughout the inference, so that design contours reflect these.

Non-stationarity with respect to covariates $\boldsymbol{X}$ is captured in the model using a Penalized Piecewise Constant (PPC) approach. Namely, covariates are split into bins considered to be roughly homogeneous. Model parameters are then estimated as constants within each covariate bin. To avoid over-fitting, a penalty on the parameter difference between covariate bins can be imposed, e.g. for the non-stationary GP scale parameter. An appropriate value for this roughness penalty is estimated using k-fold cross validation. Other parameters, such as the rate of concurrence and threshold, are simply estimated independently per covariate bin.
\\
Below we give an overview of the 5 stages of analysis included in the PPC software. Note that these stages are described here in terms of the simplest application of the software, namely using one conditioned and one associated response varying with respect to a single covariate. The approach extends easily to a higher number of associated responses and covariates, and indeed the software can be used for such analysis. 
\begin{enumerate}
\item The first stage deals with finding or simulating storm peaks. Peaks are chosen using the main (conditioning) variable $\dot{Y}_1$. $\dot{Y}_2$ is then the associated value at the time of the storm peak in $\dot{Y}_1$. This is discussed in section \ref{Sec:Stg1}.
\item Secondly the user splits the data into bins based on the marginal response characteristics. This is discussed in section \ref{Sec:Stg2}.
\item The marginal PPC models for the distribution of $\dot{Y}_1$ and $\dot{Y}_2$ are then fitted in turn and are used to transform response data on the original scale, $\{\dot{y}_{1i},\dot{y}_{2i}\}_{i=1}^{N}$, to data on Gumbel or Laplace scale, $\{y_{1i},y_{2i}\}_{i=1}^{N}$. Details of this step are discussed in section \ref{Sec:Stg3}.
\item We then fit a conditional extremes model for $Y_2|Y_1$ for various choices of threshold for the conditioning variate, retaining the estimated model parameters and residuals. This is discussed in section \ref{Sec:Stg4}.
\item Finally, we estimate design contours. Using the output from the marginal and conditional extremes model, Monte Carlo simulations are run and contours drawn using various methods. Details of this step are discussed in Section \ref{Sec:Stg5}.
\end{enumerate}

This software was developed as part of a project part-funded by the European Union ERANET entitled “Environmental Contours for SAfe DEsign
of Ships and other marine structures" (ECSADES), along with a review paper on the definition and application of environmental contours \citep{RssOE19}. Further, this software is applied to analysis of surge in the Northern North Sea and described in detail in \cite{RssEA17b}. 

\noindent\textbf{List Of Symbols}
\begin{itemize}
	\item $D$ = total number of dimensions / responses Y, indexed by $d$
	\item $N$ = number of observations, i.e. length of $Y$
	\item $i$ = index on response or covariate observation ($\in \{1,\ldots,N\}$) 
	\item $C$ = total number of covariates, indexed by $c$
	\item $B$ = total number of covariate bins, indexed by $b$
	\item $\dot{Y}$ = response data on original scale
	\item $Y$ = random variable representing response data on standard (Laplace or Gumbel) scale
	\item $X$ = random variable representing covariate data 
	\item $R$ = number of different return-periods in analysis (return-value CDFs and contours)
	\item $t$ = storm-picking non-exceedance probability (Stage 1)
	\item $\tau$ = non-exceedance probability for marginal modelling (Stage 3)
	\item $\psi$ = threshold associated with non-exceedance probability $\tau$ for marginal modelling (Stage 3)
	\item $\xi, \nu$ = marginal model parameters for the generalised Pareto distribution above the threshold 
	\item $\omega, \kappa, l$ = marginal model parameters for Gamma distribution below the threshold
	\item $\lambda$ = marginal model roughness penalty on extent to which $\nu$ can vary by covariate bin (Stage 3)
	\item $\tilde{\tau}$ = non-exceedance probability for Heffernan \& Tawn modelling (Stage 4)
	\item $\phi$ = threshold associated with non-exceedance probability  $\tilde{\tau}$ for Heffernan \& Tawn modelling (Stage 4)
	\item $\alpha, \beta, \mu, \sigma$ = Heffernan \& Tawn model parameters
	\item $\tilde{\lambda}$  = Heffernan \& Tawn model roughness penalty on extent to which $\alpha$ can vary by covariate bin (Stage 4)
\end{itemize}




%-------------------------------------------------------------------------------
\section{Stage 1: Data preparation}  \label{Sec:Stg1}
% STAGE 1 _ DATA GENERATION
%%%%%%%%%%%%%%%%%%%%%
In Stage 1 we prepare $D$-dimensional peaks over threshold response data  $\{\dot{y}_{1i},\dot{y}_{2i},\ldots \dot{y}_{Di}\}_{i=1}^{N}$  with $C$ associated covariates $\{x_{1i},x_{2i},\ldots,x_{Ci}\}_{i=1}^{N}$.  The user has a choice of 2 different run files for this stage, both producing a data file called \verb|Data.mat|, the format of which is described below. When data is available,  for example time series observations for $H_S$ and $T_P$,  \verb|Stage1_PeakPicking| should be run to identify storm peak data from the sea-state observations. In the case that we have no existing data, or we simply want to test the model, \verb|Stage1_SimulateData.m| should be run to generate response data directly.

\textbf{Run Scripts}: \verb|Stage1_PeakPicking.m| OR \verb|Stage1_SimulateData.m| (in the \verb|Case1| folder)\\
\textbf{Output data}: \verb|Output\Data.mat| with the following contents
\begin{itemize}
	\item  \verb|Dat.Y| [$N \times D$] matrix of response data, with the main (conditioned) response in the first column; and associated response data in the subsequent columns.
	\item \verb|Dat.X| [$N \times C$] matrix of covariate data, with each column representing a different covariate.
	\item \verb|Dat.RspLbl| [$1 \times D$] cell array containing string descriptions of the responses, e.g. \verb|{'Hs','Tp'}|.
	\item \verb|Dat.CvrLbl| [$1 \times C$] cell array containing string descriptions of the covariates.
	\item \verb|Dat.IsPrd| [$1 \times C$] boolean vector marking periodicity of covariates (0 = non-periodic; 1 = periodic).
\end{itemize}
You can also skip Stage 1 entirely by manually populating an equivalent \verb|Data.mat| data file with a format identical to that desribed above. 
 
Figures generated when running the PPC software are stored in a \verb|Figures| subdirectory, with the stage they are generated in indicated by prefix \verb|StgX_| in the file name. 

\subsection{Stage1$\_$PeakPicking} %%%
This script converts time series data into peaks-over-threshold data, suitable for modelling with the generalised Pareto distribution. 

An example data file called \verb|CNS_ResponseData.mat| is provided with this software. If you wish to use this data file, its location should be provided to the \verb|load()| command at the start of the MATLAB script. If you are instead using a dataset of your own, not in \verb|.mat| format, you should replace this line with a call to e.g. the \verb|readcsv()| function so that you can enter your data properties to the relevant inputs described below. 


\textbf{Inputs}:
	\begin{itemize}
		\item \verb|Rsplbl| [$D\times 1$]  cell array, containing string descriptions/names for the main and associated responses (in that order) - ensures that plots produced by the analysis are labelled correctly.
		\item \verb|CvrLbl| [$C\times 1$]  cell array containing string descriptions/names for the covariate(s).
		\item \verb|Rsp| [$N\times 1$] vector containing the main response data (the response which we condition on).
		\item\verb|Cvr| [$N\times C$] matrix where each column contains different covariate data. Number of columns must match size of \verb|CvrLbl|. 
		\item \verb|IsPrdCvr| [$C\times 1$ boolean] flag dictating periodicity of covariate(s). If 1, covariate data loops on 360. When using more than one covariate, this is a vector input with one flag per covariate, e.g. [1,0]. Note that, if you have a periodic covariate which is \textit{not} on [0,360), you must rescale it to cover this range to enable periodicity to be accounted for. Non-periodic covariate data can, on the other hand, be provided on any scale. 
		\item \verb|Asc| [$N \times (D-1)$] matrix where each column contains a different associated response - the responses which will be conditioned on the value of the main response given in \verb|Rsp|. Number of columns must match the size of \verb|RspLbl| minus one. 
		\item \verb|NEP| [scalar] non-exceedance probability (on [0,1)) used to define the threshold for storm-identification.
	\end{itemize}


Suppose that we have set the main response \verb|Rsp|$ = H_{S}$ and associated response  \verb|Asc|$ = T_{P}$; the identification of storm trajectories and storm peak exceedances is illustrated in Figure \ref{fig:StormPeak}. Note that we peak pick over the main response  \verb+Rsp+ (in this case $H_{S}$) and take \emph{associated} observations as peaks over threshold for $T_{P}$. 
\begin{figure}[hb]
\centering
\includegraphics[width=0.8\textwidth]{Figures/StormPeakDiagram}
\caption{Peak Picking Illustration}
\label{fig:StormPeak}
\end{figure}

An example of storm peak-picked data is shown for North Sea data in Figures \ref{fig:Stg1_Data_Margins} and \ref{fig:Stg1_Data_Joint}. A quantile level of $t=0.6$ was used to set the peak picking threshold giving 2566 storms.

\begin{figure}
\centering
\includegraphics[width=0.9\textwidth]{"PPCGuideExamples/TpHs/Figures/Stg1_Data_Margins"}
\caption{Marginal Hs and Tp as a function of direction for North Sea data. Storm peaks shown in black, all sea states in grey} 
\label{fig:Stg1_Data_Margins}
\end{figure}

\begin{figure}
\centering
\includegraphics[width=0.9\textwidth]{"PPCGuideExamples/TpHs/Figures/Stg1_Data_Joint"}
\caption{Joint distribution Hs and Tp  for North Sea data. Storm peaks shown in black, all sea states in grey} 
\label{fig:Stg1_Data_Joint}
\end{figure}

\subsection{Stage1$\_$SimulateData}
An alternative to using real data is to test the model using simulated data with known characteristics. Note that, though this update to the code accommodates \emph{fitting} models for multivariate cases with multiple covariates; the simulation script is restricted to bivariate cases with a single covariate only. The first four inputs required by the user set the dimensions of the data to be simulated:


\begin{enumerate}
\item \verb+nDmn+ [scalar] the number of response variables you want to simulate
\item \verb+nObs+ [scalar] the number of observations you want to simulate 
\item \verb+nBin+ [scalar] the number of covariate bins you want (common to both margins if \verb+nDmn+ $> 1$)
\item \verb+BinEdg+ [$1 \times $ \verb+nBin+] vector of edges of covariate bins on $[0,360]$ (these will wrap around 0)
\end{enumerate}

\begin{figure}
\centering
\includegraphics[width=\textwidth]{"PPCGuideExamples/MvnRho09/Figures/Stg1_Data_Simulated_Margins"}
\caption{Example Simulated Data}
\label{fig:Stg1_Data_Simulated_Margins}
\end{figure}
For each response, the user is then required to set the following distributional properties based on the number of bins \verb+nBins+ you specified:
\begin{enumerate}
\item \verb+MM.Shp+ [$1 \times $ \verb+nBin+] vector of GP shape parameters for each covariate bin
\item \verb+MM.Scl+ [$1 \times $ \verb+nBin+] vector of GP scale parameters for each covariate bin
\item \verb+MM.Thr+ [$1 \times $ \verb+nBin+] vector of GP thresholds for each covariate bin 
\item	\verb+Rat+ [$1 \times $ \verb+nBin+]  vector of Poisson rate parameters for each covariate bin
\end{enumerate}
Finally, in the case that the user chooses to simulate two responses (\verb+nDmn+ $=2$), the dependence model used and its associated parameters should also be set with the following inputs:
\begin{enumerate}
\item \verb+Jnt.Mth+: Choice of dependence model: multivariate normal \verb+MVN+; logistic \verb+LGS+; or asymmetric logistic \verb+ASL+
\item Associated parameters:
   \begin{itemize}
      \item\verb+MVN+ : dependence parameter \verb+Rho+ $ \in [0,1]$
      \item \verb+LGS+ : dependence parameter \verb+Alp+  $\in [0,1]$
      \item	\verb+ASL+ : dependence parameter \verb+Alp+  as above and weighting parameters \verb+Theta+ (one for each response/margin) $\in [0,1]$ setting the proportion of `random' points off of the logistic dependence
   \end{itemize}
\end{enumerate}

The result of running this script is the \verb|Output\Data.mat| file as described in the previous section. Figure \ref{fig:Stg1_Data_Simulated_Margins} provides an example, akin to the black-dot peak observations in Figure \ref{"../PPC_Analysis/TpHs/Figures/Stg1_Data_Margins"}. 


%\vspace{20pt}
% \noindent \textbf{Dependence Structures} \\
%Simulated responses with common PPC margins and a multivariate normal dependence structure with parameter \verb+Rho+ $= 0.8$ are illustrated in Figure \ref{fig:MVN}. Figure \ref{fig:LGS} illustrates the logistic dependence structure with \verb+Alp+ $ = 0.3$. Note that the multivariate normal structure is asymptotically independent, whilst the logistic alternative is asumptotically dependent. 
%
%\begin{figure}
%    \centering
%    \begin{subfigure}{0.4\textwidth}
%        \includegraphics[width=\textwidth]{Figures/Stg1Sim/Data_Simulated_Joint_MVN0p8.jpg}
%        \cprotect\caption{multivariate normal dependence structure with  \verb+Rho+ $= 0.8$  }
%        \label{fig:MVN}
%    \end{subfigure}
%    \begin{subfigure}{0.4\textwidth}
%        \includegraphics[width=\textwidth]{Figures/Stg1Sim/Data_Simulated_Joint_LGS0p3}
%        \cprotect\caption{logistic dependence structure with \verb+Alp+ $ = 0.3$}
%        \label{fig:LGS}
%    \end{subfigure}
%    \caption{Simulated responses on common PPC margins with two different joint dependence structures}\label{fig:MvnLgsDep}
%\end{figure}
%
%The asymmetric logistic dependence option is illustrated in Figure \ref{fig:ASL}. This model is a more complex version of the logistic model which facilitates the simulation of more `physical' or realistic data. As the strength of dependence is more prevalant in higher values of Response 1, the threshold choice is particularly influential in this case. 
%
%\begin{figure}
%    \centering
%    \begin{subfigure}{0.4\textwidth}
%        \includegraphics[width=\textwidth]{Figures/Stg1Sim/Data_Simulated_Joint_ASL0p3_0p1_0p4.jpg}
%        \cprotect\caption{\verb+Alp+ $= 0.3$, \verb+Theta+ $ = [0.1,0.4]$}
%    \end{subfigure}
%    \begin{subfigure}{0.4\textwidth}
%        \includegraphics[width=\textwidth]{Figures/Stg1Sim/Data_Simulated_Joint_ASL0p3_0p5_0p8.jpg}
%	\cprotect\caption{ \verb+Theta+ $ = 0.3$,  \verb+Theta+ $= [0.5,0.8]$}
%    \end{subfigure}
%    \caption{Asymmetric logistic dependence}\label{fig:ASL}
%\end{figure}



%-------------------------------------------------------------------------------
\section{Stage 2: Choose covariate bins} \label{Sec:Stg2}
% Stage2_Guide.tex
At the start of Stage 2, peak-picked data $\{y_{di}\}_{i=1}^{N}$ for each marginal response $d \in \{1,\ldots,D\}$ are loaded from Stage 1. In order to fit a piecewise constant Gamma-GP model to this data, we first need to specify covariate bins \verb|BinEdg|. This script is used to experiment with and set covariate bin edges. A plot of the marginal data against the covariate(s) with current bin locations marked in red is produced. The goal is to set bin edges which effectively separate the data into sections with homogeneous covariate characteristics (rate and scale), after which we move onto Stage 3. As soon as you are happy with your bin choice, you can move onto Stage 3. The set of bin-edges you last tried will automatically be fed to Stage 3 for subsequent use.

\subsection{Running Stage2}
\textbf{Run Script}: \verb|Stage2_SetBinEdges.m|\\
\textbf{Output files}: \verb|Output\Bin.mat|\\
\textbf{Inputs}: \verb|BinEdg| [$ 1 \times C$] cell array storing bin edges for each covariate.

\mbox{

}

In the case of a single covariate, bin edges should be provided to input \verb|BinEdg| in $\{[...]'\}$ format. Note that we need to transpose ($'$ operation in Matlab) to put the bin edges into long vector format. In the case of multiple covariates; bin edges should be provided in $\{[...]',[...]'\}$ format; resulting in 2D bins which are the multiplicative combination of bins in each individual covariate dimension. 

Note that, for covariates identified as periodic (setting \verb|IsPrdCvr| = 1 in Stage1), bins will automatically wrap around 360. This means that, if 0 or 360 are not specified in the vector of bin edges for that covariate; by default there will be a bin which straddles 0. If your covariate data is periodic but not on $[0,360)$, you will need transform it to $[0,360)$, e.g. by adjusting the raw data within the Stage2 script, before assigning it to \verb|Cvr|. 

If you are using a non-periodic covariate, the data can be on any scale but bear in mind that the first and last entries in the bin-edge vector will be interpreted as end-points for the range of the covariate. Specifically, you should take care to ensure that the outer bin edges (first and last) are wider than the range of the data. If you do not do this, an error will be produced when a check is made that the data lies within the range defined by the first and last bin edges. 

Further points to note:
\begin{itemize}
\item A warning will be produced if you have too few observations ($<30$ total number of observations, not exceedances) in any given bin. This is to ensure you have enough data to fit to in each bin and prevents you from over-fitting by defining too many bins. 
\item If the total number of bins in the model is $>16$, the code can struggle to estimate the generalised Pareto model well, so the number of bins should be kept relatively small. Note that a small number of bins in each covariate dimension multiplies to a large number of total bins; e.g. 4 bins in direction and season results in $4\times 4 = 16$ bins in total.
\item This code is designed to run non-stationary models and hence expects some form of covariate input. A non-stationary (covariate-free) model can be run using this code however, by creating a single periodic bin via:
	\begin{itemize}
		\item supplying e.g. time or an index on the observations to the \verb|Cvr|;
		\item setting \verb|IsPrd| to 0 for all covariates (enforcing periodicity);
		\item setting \verb|BinEdg| to $\{[\min( \verb|Cvr|),\max(\verb|Cvr|)]'\}$.  
	\end{itemize} 
The user is however \textbf{strongly encouraged} to incorporate covariates which are known to strongly affect the response(s). Failure to account for covariate effects can lead to very different return-value estimates and environmental contours. 
\end{itemize}

\subsection{Outputs}
Running Stage 2 creates a MATLAB data file (\verb|.mat| extension) called \verb|Data|, stored in a folder called \verb|Output|.

The following Figures are also generated and saved as \verb|.pdf|s in the \verb|Figures| folder. Figure \ref{fig:Stg2_Data_BinEdges} illustrates the user's bin choice as red lines on top of scatter plots of response(s) against covariate(s). Then for each associated variable, a figure like Figure \ref{fig:Stg2_Data_BinScatterPlot_Y2_Y1} is produced, containing scatter plots of the associated response on the $y$ axis and main / conditioned response on the $x$ axis, broken out by covariate bin. The dependence structure illustrated by theses subplots (non-stationary with respect to covariates) is what we aim to model via marginal and conditional extremes modelling in subsequent Stages 3 and 4. 

\begin{figure}
	\centering
	\includegraphics[width=0.9\textwidth]{"./PPCGuideExamples/Tp_Hs_ByDrc/Figures/Stg2_Data_BinEdges"}
	\caption{Example bin allocation. Bins are chosen at [0,25,60,230,275,315] degrees. Storm peak data shown in black, chosen bin edges are shown with red dashed lines.}
	\label{fig:Stg2_Data_BinEdges}
\end{figure}

\begin{figure}
	\centering
	\includegraphics[width=0.9\textwidth]{"./PPCGuideExamples/Tp_Hs_ByDrc/Figures/Stg2_Data_BinScatterPlot_Y2_Y1"}
	\caption{ Scatter plots of storm peak responses broken out by bin.}
	\label{fig:Stg2_Data_BinScatterPlot_Y2_Y1}
\end{figure}



%-------------------------------------------------------------------------------
\section{Stage 3: Fit marginal PPC models} \label{Sec:Stg3}
\subsection{The penalised piecewise constant model}
%-------------------------------------------------------------------------------

%
Stage3 fits a PPC extreme value model to a single marginal response. Non-stationary marginal extreme value characteristics of each variate are estimated in turn using a Gamma-GP model (GP above the threshold, Gamma below) and roughness-penalised maximum likelihood estimation. For a given variable and covariate bin $k$, the extreme value threshold $\psi_{b}(\tau)$ is assumed to be a quantile of the Gamma distribution fitted to all data in that bin, with specified non-exceedance probability $\tau$. $\tau$ is constant across bins. Threshold $\psi_{b}(\tau)$ is estimated with no smoothing across bins. 

Threshold exceedances are assumed to follow the GP distribution with shape $\xi(\tau)$ and scale $\nu_{b}(\tau)$. Since estimation of the shape parameter is particularly problematic, the shape parameter is assumed constant (but unknown) across covariate bins. The extent to which the GP scale varies across bins is controlled by smoothness parameter $\lambda$.  Then parameters $\xi(\tau),\{\nu_{b}(\tau)\}$ are estimated using penalised (log-) likelihood optimisation, maximising the value of the likelihood given in Appendix \ref{App1}.

Data below the threshold is assumed to follow a 3-parameter Gamma distribution with parameters location $l_{b}(\tau)$, shape $\omega_{b}(\tau)$, and scale $\kappa_{b}(\tau)$, all piecewise constant with respect to covariate bins. The density and cumulative distribution function for this non-standard parametrisation of the Gamma distribution is provided in Appendix \ref{App2}. Note that the extent to which the Gamma parameters vary by bin is \emph{not} controlled by any smoothness parameter (whereas the GP scale's smoothness is controlled).

The Poisson rate of storm occurrence, GP scale and threshold vary across bins but are constant within each bin. 

For the given margin, the resulting PPC model is used to transform the response data on original scale to standard margins (Gumbel or Laplace, user-chosen in the Stage3 script) using the Probability Integral Transform (PIT) per covariate bin. Details of this procedure can be found in the Appendix. Laplace scale is generally preferred as it permits negative conditional dependence. In the presence of negative dependence, if the user wants to use the Gumbel distribution they must first flip the sign of one of the variables to define a positive-dependence problem.  

\subsection{Uncertainty quantification}
Two sources of randomness are carried through the estimation procedure. Firstly, the model is fitted for multiple bootstrap samples of the data, uncertainty in the resulting model parameters then being carried through to later modelling stages. Secondly, for each bootstrap sample, the marginal non-exceedance probability (used to establish the threshold within each covariate bin) is randomly sampled from a range provided by the user. 

\subsection{Running Stage3}\label{Sct:RunStg3}

\textbf{Run Scripts}: \verb|Stage3_FitMargin1.m|, \verb|Stage3_FitMargin2.m|\\
\textbf{Output files}: \verb|Output\MM1.mat|, \verb|Output\MM2.mat|\\
\textbf{Inputs}: The following inputs are listed in order of usage. Note that the parameters which you will most likely need to tune/play with are \verb|NEP|, \verb|CV.SmthLB| and \verb|CV.SmthLB|.
\begin{itemize}
	\item \verb|iDmn| [scalar] specify the response upon which to fit marginal model
	\item \verb|NEP| [$1 \times 2$] non-exceedance probability range, should be in $(0,1)$
	\item \verb|nB| [scalar] number of bootstrap re-samples - must use same number for each margin
	\item \verb|Yrs| [scalar] number of years the data spans
	\item \verb|RtrPrd| [$1 \times R$] vector of return periods (years)
	\item \verb|CV.CVMth| [boolean] If 0: only Cross Validate smoothness parameter for original dataset (fast); or 1: Cross Validate smoothness for every bootstrap resample (slow)
	 \item \verb|CV.nCV| [scalar] number of cross-validation groups
 	\item \verb|CV.nSmth| [scalar] number of smoothness parameter values tried in cross-validation
 	\item \verb|CV.SmthLB| [scalar] lower bound (log10) for smoothness range
 	\item \verb|CV.SmthUB| [scalar] upper bound (log10) for smoothness range
	 \item \verb|MarginType| [string] specify the standard margin scale on which the Heffernan \& Tawn model will be fitted (options are `Laplace' or `Gumbel')
\end{itemize}


This stage should be run at least twice; specifically, once for each margin. To keep track of the input settings used for each margin and to ensure you've fitted a marginal model for each response, it is good practice to keep \verb|nDmn| (=$D$) copies of the \verb|Stage3_FitMargin.m| script (e.g. as we have listed under `Run Scripts:' above). If you forget to fit a marginal model to one of your responses, you'll typically face the following error when running Stage 4: \verb|Meg should be an nDmn x 1 Marginal Model|. 

Note that the input settings for each margin can generally differ, however the number of bootstraps \verb|nB| and standard margin \verb|MarginType| must be common to all scripts, for consistency when fitting the conditional model in Stage 4. 

Since suitable exceedance thresholds are inherently difficult to specify; we recommend the user starts with a wide range for \verb|NEP|, say $[0.3,0.95]$, working down to a narrower band of thresholds based on Figure \ref{fig:Stg3_Hs_6_ThresholdStability} (more on this process below).

Details which should be considered carefully, to ensure a quality marginal fit, are contained in boxes. 

\subsection{Outputs}
Each application of the marginal model-fitting procedure (Stage 3) to variate/dimension $d$ creates a MATLAB data file (\verb|.mat| extension) called \verb|MM#| (with dimension $d$ in the place of \verb|#|) and stored in a folder called \verb|Output|. Before you move onto running Stage 4, you should verify that you have one such file for each dimension. The file contains the following data:

\begin{itemize}
	\item \verb|MM.X| [$N \times C$]  the covariate data
	\item \verb|MM.Y|[$N \times 1$] the observational data
	\item \verb|MM.Yrs| [scalar] the number of years of data
	\item \verb|MM.RspLbl| [string] the label for the response modelled (used in plots)
	\item \verb|MM.RspSavLbl|  [string] the label for the response modelled (used in saving files)
	\item \verb|MM.CvrLbl|  [string] the covariate labels
	\item \verb|MM.nBoot| [scalar]   number of bootstraps used
	\item \verb|MM.RtrPrd| [$1 \times R$] return periods
	\item \verb|MM.Bn| covariate bin structure (created in Stage 2)
	\item \verb|MM.Scl|  [$B \times $ \verb|nBoot|] Generalised Pareto Scale parameter 
	\item \verb|MM.Shp|  [\verb|nBoot| $ \times 1$] Generalised Pareto shape parameter 
	\item \verb|MM.Omg|  [$B \times $ \verb|nBoot|] Gamma parameter 
	\item \verb|MM.Kpp|  [$B \times $ \verb|nBoot|] Gamma parameter 
	\item \verb|MM.GmmLct| [$B \times 1$] Gamma location parameter
	\item \verb|MM.NEP| [$B \times $ \verb|nBoot|] non exceedance probability
	\item \verb|MM.Thr| [$B \times $ \verb|nBoot|] exceedance threshold 
	\item \verb|MM.Rat|  [$B \times $ \verb|nBoot|] Rate of occurrence 
	\item \verb|MM.BSInd|  [$N \times $ \verb|nBoot|] index vector for bootstrap reordering
	\item \verb|MM.nCvr| [scalar] number of covariates in the model
	\item \verb|MM.nDat| [scalar]  number of observations
	\item \verb|MM.nRtr| [scalar] number of return values $R$
	\item \verb|MM.RVPrb| [$(B+1) \times$ \verb|nRVX| $\times R$] return value probabilities CDF the final bin is the Omni return value CDF
	\item \verb|MM.RVX |  [\verb|nRVX| $ \times 1$] location at which return value CDF has been computed
	\item \verb|MM.RVMed|  [$(B+1) \times R$] median return value in each bin (plus omni)
	\item \verb|MM.nRVX|  [scalar] number of points at which return value has been computed
	\item \verb|MM.MarginType| [string] distribution used to transform to standard margins
\end{itemize}


The following figures illustrate the result of PPC model fitting for the North Sea example on the $Hs$ margin.

The leftmost panel of Figure \ref{fig:Stg3_Hs_1_DataTransform} shows the original response data plotted against covariate $\boldsymbol{X}$. The blue lines represent a 95\% confidence interval on the location of the threshold and incorporate two sources of randomness originating from bootstrap re-sampling and from drawing non-exceedance probability $\tau$ at random from a uniform distribution over range \verb|NEP|. The solid blue line indicates the threshold used for the original (not bootstrap re-sampled) dataset with $\tau$ taken to be the median of all NEPs sampled in range \verb|NEP|. The central and rightmost panels illustrate the transformation of the original dataset to uniform and then Gumbel margins (the process followed using the PIT). 

\vspace{10pt}
\noindent\fbox{%
	\parbox{\textwidth}{%
	Any inhomogeneity with respect to direction in the central plot in Figure \ref{fig:Stg3_Hs_1_DataTransform} suggests that the marginal model has not fitted well. In this case, you should adjust the bin-edges (and possibly NEP) to improve your representation of non-stationarity with respect to the covariate(s).
	}%
}
\vspace{10pt}

\begin{figure}[h]
	\centering
	\includegraphics[width=\textwidth]{"./PPCGuideExamples/Tp_Hs_ByDrc/Figures/Stg3_Hs_1_DataTransform"}
	\caption{Left panel shows sea-state data in grey and storm peaks in black. Bin edges are indicated by dashed red lines. 2.5, 50 and 97.5 percentiles of estimated threshold across all bootstraps are plotted with blue lines. The Gamma location parameter is plotted in solid red. Middle and right panels show data transformed to uniform and Laplace scale.}
	\label{fig:Stg3_Hs_1_DataTransform}
\end{figure}

Figures \ref{fig:Stg3_Hs_2_ParametersScale} and \ref{fig:Stg3_Hs_2_ParametersShape} include 95\% confidence intervals on the non-stationary GP scale and stationary GP shape parameters respectively, as a function of the covariate. Again, these are based on bootstrap resampling uncertainty and NEP sampling uncertainty. Note that empty bins will still be assigned GP parameter; in the composite likelihood the empty bin will contribute no information but global values will result. 

\begin{figure}
	\centering
	\includegraphics[width=0.9\textwidth]{"./PPCGuideExamples/Tp_Hs_ByDrc/Figures/Stg3_Hs_3_ParametersShape"}
	\caption{Black lines show  2.5, 50 and 97.5 percentiles of GP shape as a function of direction. Red lines show bin edges.}
	\label{fig:Stg3_Hs_2_ParametersScale}
\end{figure}

\begin{figure}
	\centering
	\includegraphics[width=0.9\textwidth]{"./PPCGuideExamples/Tp_Hs_ByDrc/Figures/Stg3_Hs_3_ParametersShape"}
	\caption{Black lines show  2.5, 50 and 97.5 percentiles of GP shape as a function of direction. Shape parameter is constant w.r.t to covariate.}
	\label{fig:Stg3_Hs_2_ParametersShape}
\end{figure}

Figure \ref{fig:Stg3_Hs_3_CV} illustrates the cross-validation on roughness penalty $\lambda$, via a lack-of-fit plot for values within range $[\verb|CV.SmthLB|,\verb|CV.SmthUB|]$.
 
\vspace{10pt}
\noindent\fbox{%
	\parbox{\textwidth}{%
		If the red line, indicating the optimal choice of $\lambda$ is at the left or rightmost edge of Figure \ref{fig:Stg3_Hs_3_CV}; we have not considered a wide-enough range of roughness penalty values. In this case the range of penalty values considered should be widened by adjusting input CV.SmthLB or CV.SmthUB.
	}%
}
\vspace{10pt}


\begin{figure}
	\centering
	\includegraphics[width=0.9\textwidth]{"./PPCGuideExamples/Tp_Hs_ByDrc/Figures/Stg3_Hs_4_CV"}
	\caption{Cross Validation plot showing lack of fit against chosen smoothness $\lambda$ of GP scale. Low indicates good prediction performance. The red line indicates the optimal choice.}
	\label{fig:Stg3_Hs_3_CV}
\end{figure}

\vspace{10pt}
\noindent\fbox{%
	\parbox{\textwidth}{%
	The quality of model fit within each covariate bin can be assessed using Figure \ref{fig:Stg3_Hs_4_SectorGoodnessOfFit}. Red dots outside the confidence limits of the model (plotted in black) indicate a poor fit, in which case the user might reconsider their bin choice and/or NEP range etc.
	}%
}
\vspace{10pt}

Empty bins (after thresholding) are indicated by an empty plot window for the associated sector. Figure \ref{fig:Stg3_Hs_5_OverallGoodnessOfFit} illustrates the overall goodness of fit.

\begin{figure}
	\centering
	\includegraphics[width=0.9\textwidth]{"./PPCGuideExamples/Tp_Hs_ByDrc/Figures/Stg3_Hs_5_SectorGoodnessOfFit"}
	\caption{Diagnostic for quality of model-fit, broken out by covariate (here, directional) sector. Red dots show storm peaks, black lines are 2.5, 50 and 97.5 percentiles of model prediction over bootstraps. Red dots inside the confidence limits of the model indicate good fit.}
	\label{fig:Stg3_Hs_4_SectorGoodnessOfFit}
\end{figure}

\begin{figure}
	\centering
	\includegraphics[width=0.9\textwidth]{"./PPCGuideExamples/Tp_Hs_ByDrc/Figures/Stg3_Hs_6_OverallGoodnessOfFit"}
	\caption{Diagnostic for overall quality of model-fit. Red dots show storm peaks, black lines are 2.5, 50 and 97.5 percentiles of model prediction over bootstraps. Red dots inside the confidence limits of the model indicate good fit.}
	\label{fig:Stg3_Hs_5_OverallGoodnessOfFit}
\end{figure}

\vspace{10pt}
\noindent\fbox{%
	\parbox{\textwidth}{%
		Figure \ref{fig:Stg3_Hs_6_ThresholdStability} is a key output plot, showing how the estimated GP shape parameter varies as a function of the non-exceedance probability. This plot should be used to narrow down on an NEP range over which the GP shape is relatively stable.  In the left panel a reasonable choice might be [0.5,0.75], in the right panel a narrower range might be chosen, say [0.5, 0.65]. The right limit can usually be chosen as the last point before which the confidence interval widens or there is a change in slope. The lower limit should generally not be below the mode of the data since we are fitting a tail model. We choose to use an ensemble of thresholds in our analysis in recognition of the challenge of threshold selection in extreme value statistics.
	}%
}
\vspace{10pt}

\begin{figure}
%\begin{figure}
    \centering
    \begin{subfigure}{0.45\textwidth}
	\centering
	\includegraphics[width=\textwidth]{"./PPCGuideExamples/Tp_Hs_ByDrc/Figures/Stg3_Hs_7_ThresholdStability"}
	    \end{subfigure}
    \begin{subfigure}{0.45\textwidth}
		\includegraphics[width=\textwidth]{"./PPCGuideExamples/Tp_Hs_ByDrc/Figures/Stg3_Tp_7_ThresholdStability"}
    \end{subfigure}
	\caption{GP shape $\xi$ as a function of the NEP for 2 responses Hs (left plot) and Tp (right plot) from the North Sea data. Black dots show individual bootstrap estimates, red lines are local binned median, 2.5 and 97.5 percentile estimates. A well behaved model should be stable over a range of NEP's.}
	\label{fig:Stg3_Hs_6_ThresholdStability}
\end{figure}

Finally, Figure \ref{fig:Stg3_Hs_7_ReturnValueCDF} provides 10 and 100 year return level cumulative distribution functions for each covariate (here, directional) sector. When there are fewer colours in the plot than the legend; one or more of the CDFs overlap. Empty sectors are listed in the legend with an ``Empty Bin'' description and do not have an associated CDF curve. 

\begin{figure}
	\centering
	\includegraphics[width=0.9\textwidth]{"./PPCGuideExamples/Tp_Hs_ByDrc/Figures/Stg3_Hs_8_ReturnValueCDF"}
	\caption{Marginal 10 year (upper plot) and 100 year (lower plot) return value CDFs ($Hs$). Directional sectors are show using coloured lines. The black line shows the omni-directional estimate.}
	\label{fig:Stg3_Hs_7_ReturnValueCDF}
\end{figure}
\newpage

%-------------------------------------------------------------------------------
\section{Stage 4: Fit conditional extremes model} \label{Sec:Stg4}
%-------------------------------------------------------------------------------
\subsection{Conditional extremes model}
%
The standard-scale (Gumbel or Laplace) sample $\{y_{1i},y_{2i} \ldots,y_{Di}\}_{i=1}^{N}$ above some threshold of the conditioning variate $Y_1$ is used to estimate a conditional extremes model with parameters $\boldsymbol{\alpha}_{\tilde{\tau}k},  \boldsymbol{\beta}_{\tilde{\tau}},  \boldsymbol{\mu}_{\tilde{\tau}}$ and $ \boldsymbol{\sigma}_{\tilde{\tau}}$
%
\pbe
(Y_2, Y_3 \dots Y_D| Y_1 = y_{1i}) = \boldsymbol{\alpha}_{\tilde{\tau} k} y_{1i} + y_{1i}^{\boldsymbol{\beta}_{\tilde{\tau}}} \boldsymbol{W}_{\tilde{\tau}}  \text{ for } y>\phi_{\tilde{\tau}  k}.
\pee
%
 Threshold $\phi_{\tilde{\tau} k}$ is defined as the quantile of the standard Gumbel distribution with non-exceedance probability $\tilde{\tau}$ for covariate bin $k$. Note that we have two distinct non-exceedance probabilities: one for the marginal model fitting ($\tau$) and a secondary one for the H\&T model ($\tilde{\tau}$) - the value these parameters take should not necessarily be the same. 
Parameter $\boldsymbol{\alpha} \in [0,1]$ captures the extent of positive linear dependence between the associated and conditioned variable (with a stronger positive relationship as $\alpha \rightarrow 1$). $\boldsymbol{\beta} \in (-\infty,1]$ captures the spread of data around that linear relationship, with large negative values indicating a very tight distribution of data particularly for higher values of the conditioning variable (\emph{extremal dependence}); and positive values indicating a large degree of variance around the relationship, again particularly for higher values of the conditioning variable (\emph{extremal independence}). An illustration of this varying nature of dependence which the H\&T model can capture is given in Figure \ref{fig:Stg4_HTDiagram}. 

Finally, $\boldsymbol{W}_{\tilde{\tau}}$ is a random variable with an unknown distribution, the density of which we estimate using residuals from the fitted model. For model-estimation only, we assume $\boldsymbol{W}_{\tilde{\tau}} \sim N(\boldsymbol{\mu}_{\tilde{\tau}}, \boldsymbol{\sigma}_{\tilde{\tau}}^2)$.

\begin{figure}
	\centering
	\includegraphics[width=\textwidth]{Figures/HTDiagram}
	\caption{Illustration of the impact of HT parameters $\alpha$ and $\beta$ on the structure of dependence between two standard-scale (in this case Gumbel) distributed random variables}
	\label{fig:Stg4_HTDiagram}
\end{figure}

Model-fitting therefore corresponds to estimating $\{\boldsymbol{\alpha},\boldsymbol{\beta}, \boldsymbol{\mu}, \boldsymbol{\sigma}\}$ given a sample of values for $\{Y_1, Y_2,\dots\,Y_{D}\}$. All of $\phi, \boldsymbol{\alpha}, \boldsymbol{\beta}, \boldsymbol{\mu}$ and $\boldsymbol{\sigma}$ are in principle functions of covariates. Using the conditional extremes model, we simulate joint extremes on the standard Gumbel or Laplace scale, and transform these realisations to the original scale using the probability integral transform once more. 

\subsection{Running Stage4}

Running Stage4 fits a Hefferenan and Tawn conditional extreme value model. Marginal model data and parameters (\verb|Output\MM1.mat| and \verb|Output\MM2.mat|) are loaded from stage 3.\\

\textbf{Run Scripts}: \verb|Stage4_FitHeffernanTawn.m|\\
\textbf{Output files}: \verb|Output\HT.mat|\\
\textbf{Inputs}: \\
\begin{itemize}	
	\item \verb|HTNEP| [scalar] conditional non exceedence probability range, make sure $>\exp(-\exp(-0))=0.368$ or the Gumbel transformation will fail;
	\item \verb|NonStationary| [boolean] If 0: use a stationary $\alpha$ parameter in H\&T model; if 1: fit penalised-piecewise-constant (non-stationary) $\alpha$ using the same bins as the marginal analysis;
	\item \verb|CV.CVMth| [boolean] If 0: Only cross validate roughness penalty for original dataset (fast); or 1: Cross Validate smoothness for every bootstrap re-sample (slow);
	 \item \verb|CV.nCV| [scalar] number of cross validation groups;
 	\item \verb|CV.nSmth| [scalar] number of roughnesses tried in CV;
 	\item \verb|CV.SmthLB| [scalar] lower bound (log10) for set of candidate smoothness penalties;
 	\item \verb|CV.SmthUB|  [scalar] upper bound (log10) or set of candidate smoothness penalties;
 	\item \verb|SampleLocalResid| [boolean] If this is set to true (or 1), when simulating under H\&T model residuals are resampled locally (from the current covariate bin); if false (or 0), residuals are sampled globally, i.e. from any bin. 
\end{itemize}

Note that sampling residuals locally (setting \verb|SampleLocalResid| = true) effectively gives a non-stationarity to the residual part of the H\&T model, thus improving the fit/simulation procedure. That said, in the presence of bins with very few observations, we advise that this functionality is turned off (set to false) as the simulated data for a bin with very few observations will come from resampling a very small set of residuals many times. In this case it is therefore better to sample globally to increase the number of residuals from which the simulation resamples. 

The number of bootstrap resamples is inherited from the marginal model settings. For this stage to run successfully, you must have used the same number of bootstraps in each marginal model run (already highlighted in Section{Sct:RunStg3}). 

\subsection{Outputs}
Running Stage 4 produces a MATLAB data file called \verb|HT.mat| with the following contents:
\begin{itemize}
	\item \verb|HT.Prm| [$nPrm \times (D-1) \times$ \verb|nBoot|] H\&T model parameters
	\item \verb|HT.Rsd| [$nB  \times  1$] cell array of residuals sampled for each bootstrap
	\item \verb|HT.Thr| [$nB  \times  D-1$] H\&T threshold used
	\item \verb|HT.NEP|  [$nB  \times  1$] non-exceedence probability range on (0,1)
	\item \verb|HT.nBoot| [scalar] number of bootstraps
	\item \verb|HT.X| [$N \times (D-1)  \times$ \verb|nBoot|] conditioned variable transformed to standard scale
	\item \verb|HT.Y| [$N \times (D-1)  \times$ \verb|nBoot|] associated variable transformed to standard scale
	\item \verb|HT.RV.X_Stn|: $[(B+1 )\times $ \verb|RV.nRls| $ \times R$] simulated return values for conditioned value on standard scale
	\item \verb|HT.RV.X|: [$(B +1 )\times $ \verb|RV.nRls| $ \times R$]   simulated return values for conditioned value on original scale	
	\item \verb|HT.RV.Y_Stn|: [$(B +1 )\times (D-1)\times$ \verb|RV.nRls| $ \times R$] simulated return values for associated variables (conditioned on main variable) on standard scale
	\item \verb|HT.RV.Y|: [$(B +1 )\times (D-1)\times $ \verb|RV.nRls| $ \times R$] simulated return values for associated variables (conditioned on main variable) on original scale
	\item \verb|HT.RV.nRls| [scalar] number of realisations used in return value simulation
	\item \verb|HT.n| [scalar] number of data observations $N$
	\item \verb|HT.nDmn| [scalar] number of variables (main and associated) $D$
	\item \verb|HT.SmpLclRsdOn| [boolean] flag for use of residual-sampling from local bin
	\item \verb|HT.nAlp|  [scalar] number of $\alpha$ parameters in the model
	\item \verb|HT.nPrm| [scalar] total number of H\&T model parameters
	\item \verb|HT.nBin|  [scalar] number of covariate bins
	\item \verb|HT.nRtr|  [scalar] number of return periods
	\item \verb|HT.NonStat| [boolean] non-stationary $\alpha$ parameter flag
	\item \verb|HT.A| [$N \times$ \verb|nBoot|] matrix containing the bin allocation of the data samples in each bootstrap
	\item \verb|HT.RsdInd| [\verb|nBoot| $\times 1$] cell containing indices of the bootstrap samples
	\item \verb|HT.CVMth| [boolean] cross-validation method used (see associated entry in \emph{Inputs} for more detail)
	\item \verb|HT.nCV| [scalar] number of cross-validation groups
	\item \verb|HT.nSmth| [scalar] number of smoothness parameters used
	\item \verb|HT.SmthLB| [scalar] lower bound for set of candidate smoothness penalties
	\item \verb|HT.SmthUB| [scalar] upper bound for set of candidate smoothness penalties
	\item \verb|HT.SmthSet| [$1\times$ \verb|nSmth|] set of candidate smoothness penalties
	\item \verb|HT.OptSmth| [$1 \times$ \verb|nBoot|] optimal smoothness penalty resulting from cross-validation
	\item \verb|HT.CVLackOfFit| [$nSmth\times$ \verb|nBoot|] lack of fit for roughness estimation
	\item \verb|HT.MarginType| [string] margin type for transformation to standard scale
\end{itemize}

\vspace{10pt}
	\noindent\fbox{%
		\parbox{\textwidth}{%
	Figure \ref{fig:Stg4_HT_1_SmlvsData} shows a comparison of the data and a simulation from the H\&T model. If the simulated red points do not reflect the distribution of the original data in black, consider reworking the inputs to improve the model fit. 
		}%
	}
\vspace{10pt}

\begin{figure}
	\centering
	\includegraphics[width=0.9\textwidth]{"./PPCGuideExamples/Tp_Hs_ByDrc/Figures/Stg4_HT_1_SmlvsData"}
	\caption{Comparison of original data (black) and simulation from fitted H\&T model for $10\times$ period of the data (red) on standard margins (upper plot) and original margins (bottom plot). On the original scale two different spikes can be seen at the upper right corner reflecting different marginal characteristics in T$_P$ }
	\label{fig:Stg4_HT_1_SmlvsData}
\end{figure}


On the original scale, two different spikes can be seen in the upper right hand tail of the joint distribution, reflecting different marginal characteristics in T$_P$.  

\vspace{10pt}
		\noindent\fbox{%
		\parbox{\textwidth}{%
	Plots of model parameter estimates and residual distributions as a function of threshold and covariate (direction) aid assessment of model-quality. Any inhomogeneity with respect to direction in Figure \ref{fig:Stg4_HT_2_ResidualDiagnostics2} suggests that the H\&T model has not fitted well. In this case, you should adjust the H\&T NEP range and possibly return to Stage 3 to adjust the covariate bin-edges to improve your representation of non-stationarity with respect to the covariate(s).
			}%
	}
\vspace{10pt}

\begin{figure}
	\centering
	\includegraphics[width=0.9\textwidth]{"./PPCGuideExamples/Tp_Hs_ByDrc/Figures/Stg4_HT_2_ResidualDiagnostics2"}
	\caption{Diagnostic of the residuals from the H\&T fitting. Left panel shows a histogram of the residuals. Middle panel shows a normal QQ plot.  The right panel shows residuals as a function of direction. It is typical that these residuals are quite skewed (not normal), which is why they are reused in the simulation procedure.}
	\label{fig:Stg4_HT_2_ResidualDiagnostics2}
\end{figure}

Figure \ref{fig:Stg4_HT_2_ResidualDiagnostics1} summarises the same residuals, but this time not breaking out by covariate. The residual distribution in the left subplot is compared against the normal distribution in the right subplot by way of a normal QQ plot. The residual distribution will typically not be well-matched to the line $y=x$ (i.e. not be normally distributed) as assumed when fitting the conditional model. This is why we reuse the residuals in the simulation procedure. 

\begin{figure}
	\centering
	\includegraphics[width=0.9\textwidth]{"./PPCGuideExamples/Tp_Hs_ByDrc/Figures/Stg4_HT_2_ResidualDiagnostics1"}
	\caption{Diagnostic of the residuals from the H\&T fitting: residuals plotted as a function of direction. It is typical that these residuals are quite skewed (not normal), which is why they are reused in the simulation procedure.}
	\label{fig:Stg4_HT_2_ResidualDiagnostics1}
\end{figure}



Figure \ref{fig:Stg4_HT_3_Parameters} shows model parameter $\boldsymbol{\alpha}$ for the stationary case is near 1, this indicates strong dependency between large H$_S$ and T$_P$. In the non-stationary case, Figure \ref{fig:Stg4_HT_3_ParametersNonStat} shows $\boldsymbol{\alpha}$ fairly similar in most sectors but it is highly uncertain in the sector where there is no data. $\boldsymbol{\alpha}$ nearer 0 would indicate weak or no dependency. The reader is directed back to Figure \ref{fig:Stg4_HTDiagram} for an illustration of the influence of parameters $\boldsymbol{\alpha}$ and $\boldsymbol{\beta}$ on the shape of dependence.

\begin{figure}
	\centering
	\includegraphics[width=0.9\textwidth]{"./PPCGuideExamples/Tp_Hs_ByDrc/Figures/Stg4_HT_3_Parameters_Tp"}
	\caption{Histograms of the H\&T parameters over bootstrap re-samples in a non-stationary case. The parameter uncertainty captures, marginal (bootstrap and NEP) and conditional (bootstrap and H\&T NEP) uncertainty.  }
	\label{fig:Stg4_HT_3_ParametersNonStat}
\end{figure}

\begin{figure}
	\centering
	\includegraphics[width=0.9\textwidth]{"./PPCGuideExamples/Tp_Hs_ByDrc/Figures/Stg4_HT_3_Parameters_Tp_Stationary"}
	\caption{Histograms of the H\&T parameters over bootstrap re-samples in a stationary case. The parameter uncertainty captures, marginal (bootstrap and NEP) and conditional (bootstrap and H\&T NEP) uncertainty.  }
	\label{fig:Stg4_HT_3_Parameters}
\end{figure}

\vspace{10pt}
\noindent\fbox{%
	\parbox{\textwidth}{%
	The threshold stability plots in Figures \ref{fig:Stg4_HT_4_AlphaThresholdStability} and \ref{fig:Stg4_HT_4_BetaThresholdStability} are similar to those in Figure \ref{fig:Stg3_Hs_6_ThresholdStability}. These should be used in the same way as described in section \ref{Sct:RunStg3} to find a suitable range for the H\&T NEP. A range of [0.5,0.7] would seem to be a reasonable choice here.
	}%
}
\vspace{10pt}

\begin{figure}
	\centering
	\includegraphics[width=0.9\textwidth]{"./PPCGuideExamples/Tp_Hs_ByDrc/Figures/Stg4_HT_4_AlphaThresholdStability_Tp"}
	\caption{H\&T parameter $\alpha$  as a function of the H\&T NEP. Black dots show individual bootstrap estimates, red lines are local binned median, 2.5 and 97.5 percentile estimates. A well behaved model should be stable over a range of NEP's }
	\label{fig:Stg4_HT_4_AlphaThresholdStability}
\end{figure}

\begin{figure}
	\centering
	\includegraphics[width=0.9\textwidth]{"./PPCGuideExamples/Tp_Hs_ByDrc/Figures/Stg4_HT_4_BetaThresholdStability"}
	\caption{H\&T parameter $\beta$ as a function of the H\&T NEP. Black dots show individual bootstrap estimates, red lines are local binned median, 2.5 and 97.5 percentile estimates. A well behaved model should be stable over a range of NEP's}
	\label{fig:Stg4_HT_4_BetaThresholdStability}
\end{figure}

 Figure \ref{fig:Stg4_HT_6_ConditionalReturnValueCDF} shows the return value CDFs for the North Sea T$_P$$|$H$_S$ example. Here the omni directional CDF is bimodal, this is largely due to directional differences in the T$_P$ marginal distribution. Similar effects can be seen in Figure \ref{fig:Stg4_HT_1_SmlvsData}.


\begin{figure}
	\centering
	\includegraphics[width=0.9\textwidth]{"./PPCGuideExamples/Tp_Hs_ByDrc/Figures/Stg4_HT_6_ConditionalReturnValueCDF"}
	\caption{Conditional return value CDF's $p(\textrm{Tp} | Hs_{10})$ and $p(\textrm{Tp} | Hs_{100})$. Directional CDFs are shown using coloured lines. Black line shows the omni-directional estimate. In this case using the North Sea data the omni directional CDF is bimodal, this is largely due to marginal differences in Tp. Similar effects can be seen in figure \ref{fig:Stg4_HT_1_SmlvsData}}
	\label{fig:Stg4_HT_6_ConditionalReturnValueCDF}
\end{figure}

\newpage
%-------------------------------------------------------------------------------
↑\section{Stage 5: Draw contours} \label{Sec:Stg5}
Three approaches are used to estimate extreme contours using the marginal and H\&T models: constant exceedance; direct sampling (Huseby) and Heffernan \& Tawn density contours. These methods are described briefly below. The reader is directed to \cite{HslEA17} for an excellent recent review of contour methods, and \citep{RssOE19} who discuss best-practice in the application of contours. 

%General comments re: coding approach:
All the contours pass through a lock point, defined using the extreme quantile in $Y_1$ and the conditional median in $Y_j$ (for any $j \in \{2,\ldots,$D$\}$). To efficiently compute the contours, a new importance sampling method was written into the PPC software.

The \textbf{Constant Exceedance Contour} preserves the observation count in the extremal set, as illustrated in Figure \ref{fig:Stg5_ExcDiagram}. The region of simultaneously extreme $Y_1$ and $Y_j$ is captured using a quadrant, this is of course arbitrary and many different shapes could be considered, e.g, tangent, half plane, etc. 
\begin{itemize}
\item Downside: convexity means contour will never come back on itself - pushing the upper part of the curve into a table top. The resulting curve is not very practical;
\item Upside: definition is sound in the probabilistic sense, the other contour methods do not preserve probabilities in this way. From a risk point of view it can therefore be deemed to be robust.
\end{itemize}

The \noindent\textbf{Huseby Contour} is a convex contour based on the work of \cite{HsbEA15a}. This is similar to the constant exceedance contour except that a tangential set is used instead of the quadrant. The contour is computed in angular space around a centre point. Fast changing parts of the contour can come out `spiky' so we also apply a moving average to try to overcome some of these issues. 
\begin{itemize}
	\item Downside: The convexity doesn't behave well in multimodel cases or other cases where the data is non-convex;
	\item Upside: A complete contour is produced - covering the full angular space around its centre point, not just the region with the largest responses. 
\end{itemize}

The \noindent\textbf{Hefferenan and Tawn density} contour is defined by gridding the data on the original scale and calculating the density of simulations in each bin, and then drawing the line which preserves the density of the bin containing the lock point. Figure \ref{fig:Stg5_EmpDnsDiagram} illustrates this method. The `omni' contour is computed as a weighted sum across the covariate-binned contours (weighted by rate of occurrence).
\begin{itemize}
\item Downside: the contour is non-invariant to transformations of variables;
\item Upside: we get a contour which hugs the data in the way we might expect (without a table top, and without the extra roundness sometimes induced by the convexity assumption of the Huseby contour).
\end{itemize}


\subsection{Running Stage5}
Marginal and conditional model parameters are loaded from Stage 3 and Stage 4 respectively.

\textbf{Run Scripts}: \verb|Stage5_Contour.m|\\
\textbf{Output files}: \verb|Output\Cnt.mat|\\

\textbf{Inputs}:
\begin{itemize}
\item \verb|Mth| [string] cell array of contour methods to be used:
	\begin{itemize}
		\item \verb|Exc|: constant exceedance contour
  		\item \verb|HTDns|: constant density contour on standard margins, uses density form of H\&T to get contour        
  		\item \verb|Hus|: convex ``Direct Sampling'' contour of Huseby
	\end{itemize}
\item \verb|nSml| [scalar] number of simulations under H\&T model (upon which contours are estimated). May need to increase this when you have lots of bins, or see lack of smoothness in e.g. Huseby contour
\end{itemize}

\textbf{Output}: Contour structure \verb|Cnt|
\begin{itemize}
	\item \verb|Cnt.nPon| [scalar] how many points from which to draw contour
	\item \verb|Cnt.Rng| [\verb|nPon| $ \times (K+1) \times R$] conditioned values for contour
	\item \verb|Cnt.XY| [\verb|nMth| $ \times 1$] cell array for the contour lines. In case of Exc and Hus methods $XY(i)$ is [\verb|nPon| $\times 2 \times (B+1) \times R \times (D-1)$] defining contour lines in case of HTDns $XY(i)$ is a $[(B+1)\times R \times (D-1))]$ cell with sub-elements $[2 \times $ \verb|nPon|] defining the contour in this case \verb|nPon| varies for each contour bin, return period and associated variable. 
	\item \verb|Cnt.Mth|  [\verb|nMth| $ \times 1$], (cell array) of contour methods used
	\item \verb|Cnt.nMth|[scalar], number of contouring methods
	\item \verb|Cnt.nBin|[scalar], number of covariate bins
	\item \verb|Cnt.nLvl|[scalar], number of contour levels chosen
	\item \verb|Cnt.nAsc|[scalar] number of associated variables
	\item \verb|Cnt.Sml| structure importance sampled simulation under the model       
	\item \verb|Cnt.LvlOrg| [$(B+1) \times R \times (D-1)$], contour level on original scale of conditioned variable (lock point)
	
\end{itemize}

Figures \ref{fig:Stg5_Contour_1_Omni} and \ref{fig:Stg5_Contour_2_Binned_Tp} show comparisons of the 3 methods for the North Sea $T_P| H_S$ example. 

\begin{figure}
	\centering
	\includegraphics[width=\textwidth]{"./PPCGuideExamples/Tp_Hs_ByDrc/Figures/Stg5_Contour_1_Omni"}
	\caption{Comparison of contour methods omni directionally. Contours are for 10 and 100 year return periods. Different methods are shown using coloured lines. The software currently does not support an omni HTDns method. The green circle shows the lock point at each probability level through which all the contour methods have to pass.}
	\label{fig:Stg5_Contour_1_Omni}
\end{figure}

\begin{figure}
	\centering
	\includegraphics[width=\textwidth]{"./PPCGuideExamples/Tp_Hs_ByDrc/Figures/Stg5_Contour_2_Binned_Tp"}
	\caption{Comparison of contour methods by directional sector. Contours are for 10 and 100 year return periods. Different methods are shown using coloured lines. The green circle shows the lock point at each probability level through which all the contour methods have to pass.}
	\label{fig:Stg5_Contour_2_Binned_Tp}
\end{figure}



% ----------- Old diagrams
\begin{figure}[h]
	\centering
	\includegraphics[width=\textwidth]{Figures/ExcDiagram}
	\caption{Illustration of constant exceedance contour. Green lock point is defined using the extreme quantile in $Y_1$ and the conditional median in $Y_2$. The blue line is drawn such that, as $Y_1$ is decreased, the number of observations in the blue quadrant is preserved. The red line is drawn such that, as $Y_1$ decreases, the number of observations in the red quadrant is preserved.}
	\label{fig:Stg5_ExcDiagram}
\end{figure}


%\begin{figure}[h]
%	\centering
%	\includegraphics[width=\textwidth]{Figures/RadQntDiagram}
%	\caption{Illustration of radial quantile contour. Each radial bin has a different colour, the binned quantile estimate is then show with a black dot.}
%	\label{fig:Stg5_RadQntDiagram}
%\end{figure}

\begin{figure}[h]
	\centering
	\includegraphics[width=0.7\textwidth]{Figures/EmpDnsDiagram}
	\caption{Illustration of empirical density contour. The coloured squares represent the observation counts per bin.}
	\label{fig:Stg5_EmpDnsDiagram}
\end{figure}

\newpage


\section{Example: Multiple Associated Variables and Covariates} \label{Sec:Exm}
%-------------------------------------------------------------------------------

Having illustrated the key inputs and outputs of the code for a simple case - single associated variable and single covariate - in this section we provide a higher-dimensional example (you'll find the associated matlab run-scripts in the \verb|Case2| folder). We set the main (conditioning) response to Over-turning Moment and use two associated variables - wind speed and H$_S$. Further, we will use two covariates: wave direction (periodic) and season, non-periodic as (for illustration) we use day-of-the-year. The resulting figures below illustrate the result of running PPC with higher dimensions. 

\begin{figure}
	\centering
	\includegraphics[width=\textwidth]{"./PPCGuideExamples/HsWs_Otm_ByDrcSsn/Figures/Stg1_Data_Margins"}
	\caption{ Scatter plots of storm peak responses broken out by bin. Each row of subplots relates to a different response; and each column to a different covariate.}
	\label{fig:Ex_Stg2a}
\end{figure}

\begin{figure}
	\centering
	\includegraphics[width=\textwidth]{"./PPCGuideExamples/HsWs_Otm_ByDrcSsn/Figures/Stg2_Data_BinEdges"}
	\caption{Example bin allocation. Bins are chosen at $[25,230,275,315]\times[145,270]$ degrees in direction and season. Storm peak data shown in black, chosen bin edges are shown with red dashed lines. Each row of subplots relates to a different response; and each column to a different covariate.}
	\label{fig:Ex_Stg2b}
\end{figure}
\begin{figure}
	\centering
	\includegraphics[width=\textwidth]{"./PPCGuideExamples/HsWs_Otm_ByDrcSsn/Figures/Stg2_Data_BinScatterPlot_Y2_Y1"}
	\caption{Scatter plots of storm peak data for a \emph{single response} broken out by covariate bin. An equivalent figure file exists for each response. Each row corresponds to a fixed seasonal (or second covariate) bin, with columns corresponding to directional (or first covariate) bins. The dimension of the subplot array flexes with the number of covariates and covariate bins. }
	\label{fig:Ex_Stg2c}
\end{figure}


\begin{figure}
	\centering
	\includegraphics[width=\textwidth]{"./PPCGuideExamples/HsWs_Otm_ByDrcSsn/Figures/Stg3_Hs_1_DataTransform"}
	\caption{Left column shows sea-state data for a \emph{single response} in grey and storm peaks in black plotted against direction (top row) and season (bottom row). Bin edges are indicated by dashed red lines. 2.5, 50 and 97.5 percentiles of estimated threshold across all bootstraps are plotted with blue lines. The Gamma location parameter is plotted in solid red. The middle and right columns show the response data on uniform and standard scale, against direction (top row) and season (bottom row). An equivalent figure file exists for each response. }
	\label{fig:Ex_Stg3a}
\end{figure}
\begin{figure}
	\centering
	\includegraphics[width=0.9\textwidth]{"./PPCGuideExamples/HsWs_Otm_ByDrcSsn/Figures/Stg3_Hs_2_Parameters"}
	\caption{Black lines show  2.5, 50 and 97.5 percentiles of GP and Gamma parameters as a function of direction (first row) and season (second row). Red lines show bin edges. Each column refers to a different parameter: GP scale; Gamma shape and Gamma scale. An equivalent figure file exists for each response.}
	\label{fig:Ex_Stg3b}
\end{figure}
\begin{figure}
	\centering
		\includegraphics[width=0.9\textwidth]{"./PPCGuideExamples/HsWs_Otm_ByDrcSsn/Figures/Stg3_Hs_5_SectorGoodnessOfFit"}
		
			\caption{Diagnostic for quality of model-fit, broken out by covariate bin. Red dots show storm peaks, black lines are 2.5, 50 and 97.5 percentiles of model prediction over bootstraps. Red dots inside the confidence limits of the model indicate good fit. Each row corresponds to a fixed seasonal (or second covariate) bin, with columns corresponding to directional (or first covariate) bins. The dimension of the subplot array flexes with the number of covariates and covariate bins. }
	
		\label{fig:Ex_Stg3c}
	\end{figure}

\begin{figure}
	\centering
	\includegraphics[width=0.9\textwidth]{"./PPCGuideExamples/HsWs_Otm_ByDrcSsn/Figures/Stg4_HT_1_SmlvsData"}
	\caption{Comparison of original data (black) and simulation from fitted H\&T model (red) for $10 \times$ period of the data (red) on standard margins (upper row) and original margins (bottom row). Each column relates to a different associated response conditioned on the main response; specifically, column $d$ relates to $Y_d | Y_1$ for $d \in \{2,...,D\}$. }
	\label{fig:Ex_Stg4a}
\end{figure}

\begin{figure}
	\centering
	\includegraphics[width=0.9\textwidth]{"./PPCGuideExamples/HsWs_Otm_ByDrcSsn/Figures/Stg4_HT_2_ResidualDiagnostics1"}
	\caption{Diagnostic of the residuals from H\&T model fitting, where each row of plots corresponds to a different associated variable. Column 1 contains histograms of the residuals. The second column compares this distribution with the normal distribution (the form assumed in the model) via a normal QQ plot. It is typical that these residuals are quite skewed (not normal), which is why they are reused in the simulation procedure.}
	\label{fig:Ex_Stg4b1}
\end{figure}

\begin{figure}
	\centering
	\includegraphics[width=0.9\textwidth]{"./PPCGuideExamples/HsWs_Otm_ByDrcSsn/Figures/Stg4_HT_2_ResidualDiagnostics2"}
	\caption{Diagnostic of the residuals from H\&T model fitting: residuals plotted against covariate(s). The $d$'th row of subplots corresponds to the model $Y_d | Y_1$ for $d \in \{2,...,D\}$. Each column shows the residuals plotted against a different covariate. Inhomogeneity of residuals with respect to a covariate (assessed using this plot) points to a model which does not sufficiently account for covariate effects, in which case the user should reconsider their covariate bin choice and/or NEP choice(s).}
	\label{fig:Ex_Stg4b2}
\end{figure}

\begin{figure}
	\centering
	\includegraphics[width=0.9\textwidth]{"./PPCGuideExamples/HsWs_Otm_ByDrcSsn/Figures/Stg4_HT_3_Parameters_WS"}
	\caption{Plots of H\&T parameters over bootstrap re-samples for the model WS | H$_S$. In this case a non-stationary H\&T model was fitted, meaning that the first column contains plots of the $\alpha$ parameter as a function of the 2 covariates, direction and season. The second column contains histograms of the H\&T parameters which do not vary by covariate bin. The parameter uncertainty captures marginal (bootstrap and NEP) and conditional (bootstrap and H\&T NEP) uncertainty. An equivalent figure file exists for each of the $D-1$ associated responses. }
	\label{fig:Ex_Stg4c}
\end{figure}

\begin{figure}
	\centering
	\includegraphics[width=0.9\textwidth]{"./PPCGuideExamples/HsWs_Otm_ByDrcSsn/Figures/Stg4_HT_4_AlphaThresholdStability_WS"}
	\caption{H\&T parameter $\boldsymbol{\alpha}$  as a function of the H\&T NEP for the model WS | Otm.. Black dots show individual bootstrap estimates, red lines are local binned median, 2.5 and 97.5 percentile estimates. A well behaved model should be stable over a range of NEP's. The dimension of the subplot array flexes with the number of covariates and covariate bins. An equivalent figure file exists for each of the $D-1$ associated responses.}
	\label{fig:Ex_Stg4d}
\end{figure}

\begin{figure}
	\centering
	\includegraphics[width=0.9\textwidth]{"./PPCGuideExamples/HsWs_Otm_ByDrcSsn/Figures/Stg4_HT_5_SectorGoodnessOfFit"}
	\caption{Cross Validation plot showing lack of fit against chosen smoothness $\nu$ of H\&T parameter $\boldsymbol{\alpha}$. Low indicates good predictive performance. The red line indicates the optimal choice. When the red line is at the edge of the $\nu$ domain, the user should extend the range of $\nu$ considered using inputs HT.SmthLB and HT.SmthUB.}
	\label{fig:Ex_Stg4e}
\end{figure}

\begin{figure}
	\centering
	\includegraphics[width=0.9\textwidth]{"./PPCGuideExamples/HsWs_Otm_ByDrcSsn/Figures/Stg4_HT_6_ConditionalReturnValueCDF"}
	\caption{Conditional 10 (first column) and 100 (second columns) return value CDF's of $p(H_S| \textrm{Otm})$ and  $p(WS | \textrm{Otm})$. CDFs for each covariate bin are shown using coloured lines. Black line shows the omni-directional estimate.}
	\label{fig:Ex_Stg4f}
\end{figure}

\begin{figure}
	\centering
	\includegraphics[width=0.9\textwidth]{"./PPCGuideExamples/HsWs_Otm_ByDrcSsn/Figures/Stg5_Contour_2_Binnned_WS"}
	\caption{Contour plots for a single associated response (conditioned on the main response), in this case WS | Otm, broken out by covariate bin. Two contour lines are drawn: one for 10 year and one for the 100 year return period. Each coloured line corresponds to a different contouring method. The green circle shows the lock point at each probability level through which all the contour methods have to pass. The dimension of the subplot array flexes with the number of covariates and covariate bins. An equivalent figure file exists for each of the $D-1$ associated responses. }
	\label{fig:Ex_Stg5}
\end{figure}

\newpage


%\section{Case studies}
%
\subsection{Case Study 1: Simulation}

%% \subsection Case Study 1: MVN 0.9

\subsubsection{Simulated Data}
\begin{figure}
	\centering
	\includegraphics[width=\textwidth]{"../PPC_Analysis/MvnRho09/Figures/Stg2_Data_BinEdges"}
	\caption{Simulation: Main Response $Y_{1}$ (upper plot), Associated Response $Y_{2}$ (lower plot)}
	\label{fig:CaseMvnSim}
\end{figure}

\begin{figure}
	\centering
	\includegraphics[width=\textwidth]{"../PPC_Analysis/MvnRho09/Figures/Stg1_Data_Simulated_Joint"}
	\caption{Simulation: Joint Associated $Y_{2}$}
	\label{fig:CaseMvnSimJnt}
\end{figure}

\subsubsection{Marginal GP Fit to Main Response $Y_{1}$)}

\begin{figure}
	\centering
	\includegraphics[width=\textwidth]{"../PPC_Analysis/MvnRho09/Figures/Stg3_Response_1_DataTransform"}
	\caption{Associated $Y_{2}$: Transformation to Gumbel Margins}
	\label{fig:CaseMvnMrg1DatTrns}
\end{figure}

\begin{figure}
	\centering
	\includegraphics[width=\textwidth]{"../PPC_Analysis/MvnRho09/Figures/Stg3_Response_2_Parameters"}
	\caption{Associated $Y_{2}$: Fitted GP Parameters}
	\label{fig:CaseMvnMrg1Prm}
\end{figure}

%% Sub plot from here...
\begin{figure}
	\centering
	\includegraphics[width=\textwidth]{"../PPC_Analysis/MvnRho09/Figures/Stg3_Response_3_CV"}
	\caption{Associated $Y_{2}$: Cross-Validation Lack of Fit}
	\label{fig:CaseMvnMrg1CV}
\end{figure}

\begin{figure}
	\centering
	\includegraphics[width=\textwidth]{"../PPC_Analysis/MvnRho09/Figures/Stg3_Response_6_ThresholdStability"}
	\caption{Associated $Y_{2}$: Parameter Stability by NEP}
	\label{fig:CaseMvnMrg1Thr}
\end{figure}


\begin{figure}
	\centering
	\includegraphics[width=\textwidth]{"../PPC_Analysis/MvnRho09/Figures/Stg3_Associated_7_ReturnValueCDF"}
	\caption{Associated $Y_{2}$: Return Value CDF}
	\label{fig:CaseMvnMrg1RV}
\end{figure}



\subsubsection{Marginal GP Fit to Associated Response $Y_{2}$}




\begin{figure}
	\centering
	\includegraphics[width=\textwidth]{"../PPC_Analysis/MvnRho09/Figures/Stg3_Associated_1_DataTransform"}
	\caption{Associated $Y_{2}$: Transformation to Gumbel Margins}
	\label{fig:CaseMvnMrg2DatTrns}
\end{figure}

\begin{figure}
	\centering
	\includegraphics[width=\textwidth]{"../PPC_Analysis/MvnRho09/Figures/Stg3_Associated_2_Parameters"}
	\caption{Associated $Y_{2}$: Fitted GP Parameters}
	\label{fig:CaseMvnMrg2Prm}
\end{figure}

%% Sub plot from here...
\begin{figure}
	\centering
	\includegraphics[width=\textwidth]{"../PPC_Analysis/MvnRho09/Figures/Stg3_Associated_3_CV"}
	\caption{Associated $Y_{2}$: Cross-Validation Lack of Fit}
	\label{fig:CaseMvnMrg2CV}
\end{figure}

\begin{figure}
	\centering
	\includegraphics[width=\textwidth]{"../PPC_Analysis/MvnRho09/Figures/Stg3_Associated_6_ThresholdStability"}
	\caption{Associated $Y_{2}$: Parameter Stability by NEP}
	\label{fig:CaseMvnMrg2Thr}
\end{figure}


\begin{figure}
	\centering
	\includegraphics[width=\textwidth]{"../PPC_Analysis/MvnRho09/Figures/Stg3_Associated_7_ReturnValueCDF"}
	\caption{Associated $Y_{2}$: Return Value CDF}
	\label{fig:CaseMvnMrg2RV}
\end{figure}




\subsection{Case study 2. Associated $T_P$ given extreme $Hs$}


\subsection{Case study 3. Associated extreme Maximum Vessel Offset given extreme $Hs$}

\subsection{Case study 4. Associated extreme Vessel Roll given extreme $Hs$}

%\subsection{Case study 4. Wind speed current speed? }



%-------------------------------------------------------------------------------
%\section{Discussion}
%\input{Discussion.tex}
%
%%-------------------------------------------------------------------------------
\appendix
%The details of the model can be found in Appendix \ref{App1}.   

\section{Appendix 1 : Model details} \label{App1}

%
This section contains details of the underlying statistical models used. 

%Stuff to do with both responses + a covariate
 Consider a sample $\{ \dot{y}_{1i},\dot{y}_{2i},\ldots \dot{y}_{Di}\}_{i=1}^{D}$ of $D$ joint values of peaks over threshold $\dot{Y}_1$ for a conditioning variate, and a number of associated conditioned variates $\dot{Y}_j$ (where $j \in \{1,\ldots,D\}$). Further, let $\{\boldsymbol{x}_i\}_{i=1}^{N}$ be the values of an associated covariate on some covariate domain $\mathcal{X}$.
 
 Each sample vector $\{ \dot{y}_{1i},\dot{y}_{2i},\ldots,\dot{y}_{Di}\}$ is allocated to one of $B$ covariate bins (indexed using $b$) by means of an allocation vector $A$.  All joint-observations with the same covariate bin $b$ are assumed to have common extreme value characteristics.  The dependence structure between the variates on a transformed standard scale is then estimated using a conditional extremes model.
 
 Non-stationarity with respect to covariate bins is captured via a GP shape and H\&T $\alpha$ parameter which can vary across bins. The extent of variation is controlled via a roughness-penalty in the marginal and conditional extreme value likelihood functions. Note that the spatial location of bins in multiple covariates (i.e. their relative location on a grid) does not factor into the roughness-penalty. For example, a case with $4$ directional bins and $3$ seasonal bins still boils down to a likelihood which is a product over the total $B = 4\times 3 = 12$ bins. 
%
%-------------------------------------------------------------------------------
\subsection{Marginal model}


For variate $d \in \{1,\ldots,D\}$ with response data $\{\dot{y}_{di}\}_{i=1}^{n}$ and \emph{marginal} non-exceedence probability $\tau$, the likelihood function for the marginal PPC model is as follows: 
%
\[
\mathcal{L}(\tau) = \prod_{b=1}^{B}  \prod_{i;A(i)=b} 
\left\{ \begin{array}{ll}
	 \frac{1}{\nu_{b}} \left(1 + \frac{\xi}{\nu_{b}} \left(\yd_{di}-\psi_{b}\right) \right)^{-1/\xi-1} & \mbox{if $\yd_{di}>\psi_{b}(\tau)$};\\
	\frac{1}{\Gamma(\omega)\left(\frac{\kappa}{\omega}\right)^{\omega}} (\dot{y}_{di} - l_{b})^{\omega - 1} \exp \left(\frac{-\omega(\dot{y}_{di}-l_{b})}{\kappa} \right)   & \mbox{if $\yd_{di}\leq\psi_{b}(\tau)$},
		\end{array} \right. 
\]
where the upper equation is simply the Generalised Pareto density; and the lower is the 3-parameter Gamma density corresponding to the models used above and below the threshold $\psi$. The first product is over covariate bins and the second is over observations within the given covariate bin. 



Note that the threshold $\psi_{b}$ in bin $b$ is directly controlled by the non-exceedance probability $\tau$ and hence appears as a function $\psi_{b}(\tau)$. With this threshold dictating the subset of response data included in likelihood, the other parameters are also dependent on the choice of $\tau$ (and are presented as such in Section \ref{Sec:Stg3}). Conditioned on the choice of $\tau$ however, the maximum likelihood estimates of the parameters are fixed. For ease of presentation we therefore do not represent the remaining parameters as functions of $\tau$ here.  

Recall from Section \ref{Sec:Stg3} that we have an additional parameter, the smoothness penalty $\lambda$, which controls the extent to which the GP shape $\nu_{b}$ can vary across bins. Its optimal value, $\hat{\lambda}$, is chosen to maximise predictive performance using a k-fold cross-validation procedure, resulting in the following roughness-penalised negative log-likelihood function: 

\pbe
\ell(\tau, \lambda) = -\log{\mathcal{L}(\tau)} + \lambda \left( \frac{1}{B} \sum_{b=1}^B \nu_{b}^2 - \left[ \frac{1}{B} \sum_{b=1}^B  \nu_{b} \right]^2 \right).
\pee
Parameters are estimated to minimise the negative log likelihood, with the GP and Gamma parameters carried forward to subsequent inference being those associated with the optimal $\hat{\lambda}$ resulting from cross-validation.  

Note that we have likelihood functions of this form for each variate $d$, meaning that each variate will come with a different set of fitted parameters. We do not include a subscript over $d$ here for simplicity of presentation.  

 







%transform from standard to uniform margins using CDF
%case 'Gumbel'
%P = exp(-exp(-X));
%case 'Laplace'
%P = (X>0)-.5*sign(X).*exp(-abs(X));
%
%% inverse
%transform from uniform to standard margins
%using inverse CDF
%case 'Gumbel'
%X = -log(-log(P));
%case 'Laplace'
%

\subsection{Transformation to Standard Scale} \label{App2}
The marginal models are used to transform from the original scale to the standard scale (Gumbel or Laplace) using the probability integral transform. 


\begin{enumerate}
	\item First we transform data above the threshold to uniform scale using the Generalised-Pareto CDF which, with parameters $\xi, \nu$ and $\psi$, has cumulative distribution function:
	\[F_{GP}(\dot{y};\xi,\nu,\psi) = (1+\frac{\xi}{\nu} \left(\dot{y}-\psi\right) )^{-1/\xi}\]
	for $\dot{y} \in (\psi, \dot{y}^+]$  where $\dot{y}^+$ is the upper end point of the distribution. 
	Below the threshold, a gamma CDF is used: 
	\[F_{\Gamma}(\dot{y};\omega,\kappa,l) = {\frac{1}{\Gamma(\omega)}}\gamma \left(\omega,\frac{\omega(\dot{y}-l)}{\kappa }\right) \]
	where $\gamma(\cdot,\cdot)$ is the lower incomplete gamma function. Note that this an orthogonal parameterisarion taken from \cite{Cox87}.
	% Pdf: $$ f(\omega, \kappa)=(\omega)^(-\omega)/Gamma(\omega) x^(\omega-1) exp(-\omega x/\kappa)$$

	That is, we transform original scale data $\dot{y}$ to uniform scale $u$ via the following:
	\begin{align*}
	u = 
	\left\{ \begin{array}{ll}
	F_{GP}(\dot{y}) & \mbox{if $\dot{y} > \psi$};\\
	F_{\Gamma}(\dot{y}) & \mbox{if $\dot{y} \leq \psi$}.
	\end{array} \right. 
	\end{align*}
	

	\item Given uniformly distributed $u$, to transform to standard \textbf{Laplace} margins we use the inverse of the standard Laplace CDF:
	\begin{align*}
	    y = F^{-1}_{L}(u)= \text{sign}(0.5-u)\times\log(2\times \min(1-u,u)); 
	\end{align*}
	where 
	\begin{align*}
		\boldsymbol{1}(z) = 
		\left\{ \begin{array}{ll}
			1 & \mbox{if $z > 0$};\\
			0 & \mbox{if $z \leq 0$}.
		\end{array} \right. 
	\end{align*}
	For reference, this is derived from the Laplace CDF: 
	\begin{align*}
			F_{L}(z) = \boldsymbol{1}(z)-0.5\times\text{sign}(z)\times\exp(-|z|).\\ 
	\end{align*}
	To transform the uniform data $u$ to standard \textbf{Gumbel} margins we use the inverse of the standard Gumbel CDF:
	\begin{align*}
%		F_{GP}(z) = \exp(-\exp(-z))\\
		y = F^{-1}_{GP}(u)= \log(\log(-u))
	\end{align*} 
	This is derived from the Gumbel CDF: 
	\begin{align*}
		F_{GP}(z) = \exp(-\exp(-z)).\\
	\end{align*}
\end{enumerate}

Non stationary is captured by transforming each bin in turn with its associated parameters.
%-------------------------------------------------------------------------------
\subsection{Conditional model}


%
The Gumbel-scale sample $\{y_{1i},y_{2i},.., y_{Di}\}$ above some threshold $\phi$ of the conditioning variate $Y_1$ is used to estimate a stationary conditional extremes model with parameters $\boldsymbol{\alpha}, \boldsymbol{\beta}, \boldsymbol{\mu}$ and $\boldsymbol{\sigma}$
%
\pbe
(Y_2, Y_3 \dots | Y_1 = y_{1i}) = \boldsymbol{\alpha}_{k} y_{1i} + y_{1i}^{\boldsymbol{\beta}} \boldsymbol{W}   \text{ for all } i \text{ such that } y_{1i}>\phi(\tilde{\tau}),
\pee
%
where $\boldsymbol{W} \sim N(\boldsymbol{\mu}, \boldsymbol{\sigma}^2)$ is assumed for model estimation only. Threshold $\phi(\tilde{\tau})$ is defined as the quantile of the standard Gumbel distribution with H\&T non-exceedance probability $\tilde{\tau}$. Parameters are estimated to minimise the negative log-likelihood
%
\pbe
\tilde{\ell}(\tilde{\tau}) =  \sum_{d=2}^{D}\sum^B_{b=1}\sum_{\stackrel{i;b=A(i)}{y_{di}>\phi(\tilde{\tau})}} \left[ \log(2 \pi \sigma_{d}^2)+\frac{1}{2 \sigma_{d}^2} \left(y_{di} - \alpha_{db} y_{1i} - \mu_{d} y_{1i} ^{\beta_{d}}\right)^2 \right]
\pee

\pbe
\tilde{\ell}(\tilde{\tau},\tilde{\lambda}) = \tilde{\ell}(\tilde{\tau}) + \tilde{\lambda}\sum_{d=2}^{D} \left( \frac{1}{B} \sum_{b=1}^B \alpha_{db}^2 - \left[ \frac{1}{B} \sum_{b=1}^B  \alpha_{db} \right]^2 \right)
\pee
%
for each value of $\tilde{\tau}$. This is assuming a Normal distribution in the likelihood which is almost certainly not appropriate, however the distribution of $\boldsymbol{W}_{\tilde{\tau}}$ is then estimated from the sample $\{r_{di}\}$ of residuals from the fit for $d \in \{2,\ldots,D\}$:
%
\pbe
r_{di} = \frac{1}{\sigma_{d}} \left(y_{di} - \alpha_{db} y_{1i} - \mu_{d} y_{1i}^{\beta_{d}} \right) \text{ for all } i \text{ such that } y_{1i}>\phi(\tilde{\tau}) .
\pee
Below the threshold, $\phi(\tilde{\tau})$, data are re-sampled using an empirical CDF based on the ranks of the data. 

%
%-------------------------------------------------------------------------------


%-------------------------------------------------------------------------------
\bibliographystyle{plainnat}
\bibliography{phil}
%-------------------------------------------------------------------------------

%-------------------------------------------------------------------------------
\end{document}
%-------------------------------------------------------------------------------
