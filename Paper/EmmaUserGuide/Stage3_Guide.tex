\subsection{The penalised piecewise constant model}
%-------------------------------------------------------------------------------

%
Stage3 fits a PPC extreme value model to a single marginal response. Non-stationary marginal extreme value characteristics of each variate are estimated in turn using a Gamma-GP model (GP above the threshold, Gamma below) and roughness-penalised maximum likelihood estimation. For a given variable and covariate bin $k$, the extreme value threshold $\psi_{b}(\tau)$ is assumed to be a quantile of the Gamma distribution fitted to all data in that bin, with specified non-exceedance probability $\tau$. $\tau$ is constant across bins. Threshold $\psi_{b}(\tau)$ is estimated with no smoothing across bins. 

Threshold exceedances are assumed to follow the GP distribution with shape $\xi(\tau)$ and scale $\nu_{b}(\tau)$. Since estimation of the shape parameter is particularly problematic, the shape parameter is assumed constant (but unknown) across covariate bins. The extent to which the GP scale varies across bins is controlled by smoothness parameter $\lambda$.  Then parameters $\xi(\tau),\{\nu_{b}(\tau)\}$ are estimated using penalised (log-) likelihood optimisation, maximising the value of the likelihood given in Appendix \ref{App1}.

Data below the threshold is assumed to follow a 3-parameter Gamma distribution with parameters location $l_{b}(\tau)$, shape $\omega_{b}(\tau)$, and scale $\kappa_{b}(\tau)$, all piecewise constant with respect to covariate bins. The density and cumulative distribution function for this non-standard parametrisation of the Gamma distribution is provided in Appendix \ref{App2}. Note that the extent to which the Gamma parameters vary by bin is \emph{not} controlled by any smoothness parameter (whereas the GP scale's smoothness is controlled).

The Poisson rate of storm occurrence, GP scale and threshold vary across bins but are constant within each bin. 

For the given margin, the resulting PPC model is used to transform the response data on original scale to standard margins (Gumbel or Laplace, user-chosen in the Stage3 script) using the Probability Integral Transform (PIT) per covariate bin. Details of this procedure can be found in the Appendix. Laplace scale is generally preferred as it permits negative conditional dependence. In the presence of negative dependence, if the user wants to use the Gumbel distribution they must first flip the sign of one of the variables to define a positive-dependence problem.  

\subsection{Uncertainty quantification}
Two sources of randomness are carried through the estimation procedure. Firstly, the model is fitted for multiple bootstrap samples of the data, uncertainty in the resulting model parameters then being carried through to later modelling stages. Secondly, for each bootstrap sample, the marginal non-exceedance probability (used to establish the threshold within each covariate bin) is randomly sampled from a range provided by the user. 

\subsection{Running Stage3}\label{Sct:RunStg3}

\textbf{Run Scripts}: \verb|Stage3_FitMargin1.m|, \verb|Stage3_FitMargin2.m|\\
\textbf{Output files}: \verb|Output\MM1.mat|, \verb|Output\MM2.mat|\\
\textbf{Inputs}: The following inputs are listed in order of usage. Note that the parameters which you will most likely need to tune/play with are \verb|NEP|, \verb|CV.SmthLB| and \verb|CV.SmthLB|.
\begin{itemize}
	\item \verb|iDmn| [scalar] specify the response upon which to fit marginal model
	\item \verb|NEP| [$1 \times 2$] non-exceedance probability range, should be in $(0,1)$
	\item \verb|nB| [scalar] number of bootstrap re-samples - must use same number for each margin
	\item \verb|Yrs| [scalar] number of years the data spans
	\item \verb|RtrPrd| [$1 \times R$] vector of return periods (years)
	\item \verb|CV.CVMth| [boolean] If 0: only Cross Validate smoothness parameter for original dataset (fast); or 1: Cross Validate smoothness for every bootstrap resample (slow)
	 \item \verb|CV.nCV| [scalar] number of cross-validation groups
 	\item \verb|CV.nSmth| [scalar] number of smoothness parameter values tried in cross-validation
 	\item \verb|CV.SmthLB| [scalar] lower bound (log10) for smoothness range
 	\item \verb|CV.SmthUB| [scalar] upper bound (log10) for smoothness range
	 \item \verb|MarginType| [string] specify the standard margin scale on which the Heffernan \& Tawn model will be fitted (options are `Laplace' or `Gumbel')
\end{itemize}


This stage should be run at least twice; specifically, once for each margin. To keep track of the input settings used for each margin and to ensure you've fitted a marginal model for each response, it is good practice to keep \verb|nDmn| (=$D$) copies of the \verb|Stage3_FitMargin.m| script (e.g. as we have listed under `Run Scripts:' above). If you forget to fit a marginal model to one of your responses, you'll typically face the following error when running Stage 4: \verb|Meg should be an nDmn x 1 Marginal Model|. 

Note that the input settings for each margin can generally differ, however the number of bootstraps \verb|nB| and standard margin \verb|MarginType| must be common to all scripts, for consistency when fitting the conditional model in Stage 4. 

Since suitable exceedance thresholds are inherently difficult to specify; we recommend the user starts with a wide range for \verb|NEP|, say $[0.3,0.95]$, working down to a narrower band of thresholds based on Figure \ref{fig:Stg3_Hs_6_ThresholdStability} (more on this process below).

Details which should be considered carefully, to ensure a quality marginal fit, are contained in boxes. 

\subsection{Outputs}
Each application of the marginal model-fitting procedure (Stage 3) to variate/dimension $d$ creates a MATLAB data file (\verb|.mat| extension) called \verb|MM#| (with dimension $d$ in the place of \verb|#|) and stored in a folder called \verb|Output|. Before you move onto running Stage 4, you should verify that you have one such file for each dimension. The file contains the following data:

\begin{itemize}
	\item \verb|MM.X| [$N \times C$]  the covariate data
	\item \verb|MM.Y|[$N \times 1$] the observational data
	\item \verb|MM.Yrs| [scalar] the number of years of data
	\item \verb|MM.RspLbl| [string] the label for the response modelled (used in plots)
	\item \verb|MM.RspSavLbl|  [string] the label for the response modelled (used in saving files)
	\item \verb|MM.CvrLbl|  [string] the covariate labels
	\item \verb|MM.nBoot| [scalar]   number of bootstraps used
	\item \verb|MM.RtrPrd| [$1 \times R$] return periods
	\item \verb|MM.Bn| covariate bin structure (created in Stage 2)
	\item \verb|MM.Scl|  [$B \times $ \verb|nBoot|] Generalised Pareto Scale parameter 
	\item \verb|MM.Shp|  [\verb|nBoot| $ \times 1$] Generalised Pareto shape parameter 
	\item \verb|MM.Omg|  [$B \times $ \verb|nBoot|] Gamma parameter 
	\item \verb|MM.Kpp|  [$B \times $ \verb|nBoot|] Gamma parameter 
	\item \verb|MM.GmmLct| [$B \times 1$] Gamma location parameter
	\item \verb|MM.NEP| [$B \times $ \verb|nBoot|] non exceedance probability
	\item \verb|MM.Thr| [$B \times $ \verb|nBoot|] exceedance threshold 
	\item \verb|MM.Rat|  [$B \times $ \verb|nBoot|] Rate of occurrence 
	\item \verb|MM.BSInd|  [$N \times $ \verb|nBoot|] index vector for bootstrap reordering
	\item \verb|MM.nCvr| [scalar] number of covariates in the model
	\item \verb|MM.nDat| [scalar]  number of observations
	\item \verb|MM.nRtr| [scalar] number of return values $R$
	\item \verb|MM.RVPrb| [$(B+1) \times$ \verb|nRVX| $\times R$] return value probabilities CDF the final bin is the Omni return value CDF
	\item \verb|MM.RVX |  [\verb|nRVX| $ \times 1$] location at which return value CDF has been computed
	\item \verb|MM.RVMed|  [$(B+1) \times R$] median return value in each bin (plus omni)
	\item \verb|MM.nRVX|  [scalar] number of points at which return value has been computed
	\item \verb|MM.MarginType| [string] distribution used to transform to standard margins
\end{itemize}


The following figures illustrate the result of PPC model fitting for the North Sea example on the $Hs$ margin.

The leftmost panel of Figure \ref{fig:Stg3_Hs_1_DataTransform} shows the original response data plotted against covariate $\boldsymbol{X}$. The blue lines represent a 95\% confidence interval on the location of the threshold and incorporate two sources of randomness originating from bootstrap re-sampling and from drawing non-exceedance probability $\tau$ at random from a uniform distribution over range \verb|NEP|. The solid blue line indicates the threshold used for the original (not bootstrap re-sampled) dataset with $\tau$ taken to be the median of all NEPs sampled in range \verb|NEP|. The central and rightmost panels illustrate the transformation of the original dataset to uniform and then Gumbel margins (the process followed using the PIT). 

\vspace{10pt}
\noindent\fbox{%
	\parbox{\textwidth}{%
	Any inhomogeneity with respect to direction in the central plot in Figure \ref{fig:Stg3_Hs_1_DataTransform} suggests that the marginal model has not fitted well. In this case, you should adjust the bin-edges (and possibly NEP) to improve your representation of non-stationarity with respect to the covariate(s).
	}%
}
\vspace{10pt}

\begin{figure}[h]
	\centering
	\includegraphics[width=\textwidth]{"./PPCGuideExamples/Tp_Hs_ByDrc/Figures/Stg3_Hs_1_DataTransform"}
	\caption{Left panel shows sea-state data in grey and storm peaks in black. Bin edges are indicated by dashed red lines. 2.5, 50 and 97.5 percentiles of estimated threshold across all bootstraps are plotted with blue lines. The Gamma location parameter is plotted in solid red. Middle and right panels show data transformed to uniform and Laplace scale.}
	\label{fig:Stg3_Hs_1_DataTransform}
\end{figure}

Figures \ref{fig:Stg3_Hs_2_ParametersScale} and \ref{fig:Stg3_Hs_2_ParametersShape} include 95\% confidence intervals on the non-stationary GP scale and stationary GP shape parameters respectively, as a function of the covariate. Again, these are based on bootstrap resampling uncertainty and NEP sampling uncertainty. Note that empty bins will still be assigned GP parameter; in the composite likelihood the empty bin will contribute no information but global values will result. 

\begin{figure}
	\centering
	\includegraphics[width=0.9\textwidth]{"./PPCGuideExamples/Tp_Hs_ByDrc/Figures/Stg3_Hs_3_ParametersShape"}
	\caption{Black lines show  2.5, 50 and 97.5 percentiles of GP shape as a function of direction. Red lines show bin edges.}
	\label{fig:Stg3_Hs_2_ParametersScale}
\end{figure}

\begin{figure}
	\centering
	\includegraphics[width=0.9\textwidth]{"./PPCGuideExamples/Tp_Hs_ByDrc/Figures/Stg3_Hs_3_ParametersShape"}
	\caption{Black lines show  2.5, 50 and 97.5 percentiles of GP shape as a function of direction. Shape parameter is constant w.r.t to covariate.}
	\label{fig:Stg3_Hs_2_ParametersShape}
\end{figure}

Figure \ref{fig:Stg3_Hs_3_CV} illustrates the cross-validation on roughness penalty $\lambda$, via a lack-of-fit plot for values within range $[\verb|CV.SmthLB|,\verb|CV.SmthUB|]$.
 
\vspace{10pt}
\noindent\fbox{%
	\parbox{\textwidth}{%
		If the red line, indicating the optimal choice of $\lambda$ is at the left or rightmost edge of Figure \ref{fig:Stg3_Hs_3_CV}; we have not considered a wide-enough range of roughness penalty values. In this case the range of penalty values considered should be widened by adjusting input CV.SmthLB or CV.SmthUB.
	}%
}
\vspace{10pt}


\begin{figure}
	\centering
	\includegraphics[width=0.9\textwidth]{"./PPCGuideExamples/Tp_Hs_ByDrc/Figures/Stg3_Hs_4_CV"}
	\caption{Cross Validation plot showing lack of fit against chosen smoothness $\lambda$ of GP scale. Low indicates good prediction performance. The red line indicates the optimal choice.}
	\label{fig:Stg3_Hs_3_CV}
\end{figure}

\vspace{10pt}
\noindent\fbox{%
	\parbox{\textwidth}{%
	The quality of model fit within each covariate bin can be assessed using Figure \ref{fig:Stg3_Hs_4_SectorGoodnessOfFit}. Red dots outside the confidence limits of the model (plotted in black) indicate a poor fit, in which case the user might reconsider their bin choice and/or NEP range etc.
	}%
}
\vspace{10pt}

Empty bins (after thresholding) are indicated by an empty plot window for the associated sector. Figure \ref{fig:Stg3_Hs_5_OverallGoodnessOfFit} illustrates the overall goodness of fit.

\begin{figure}
	\centering
	\includegraphics[width=0.9\textwidth]{"./PPCGuideExamples/Tp_Hs_ByDrc/Figures/Stg3_Hs_5_SectorGoodnessOfFit"}
	\caption{Diagnostic for quality of model-fit, broken out by covariate (here, directional) sector. Red dots show storm peaks, black lines are 2.5, 50 and 97.5 percentiles of model prediction over bootstraps. Red dots inside the confidence limits of the model indicate good fit.}
	\label{fig:Stg3_Hs_4_SectorGoodnessOfFit}
\end{figure}

\begin{figure}
	\centering
	\includegraphics[width=0.9\textwidth]{"./PPCGuideExamples/Tp_Hs_ByDrc/Figures/Stg3_Hs_6_OverallGoodnessOfFit"}
	\caption{Diagnostic for overall quality of model-fit. Red dots show storm peaks, black lines are 2.5, 50 and 97.5 percentiles of model prediction over bootstraps. Red dots inside the confidence limits of the model indicate good fit.}
	\label{fig:Stg3_Hs_5_OverallGoodnessOfFit}
\end{figure}

\vspace{10pt}
\noindent\fbox{%
	\parbox{\textwidth}{%
		Figure \ref{fig:Stg3_Hs_6_ThresholdStability} is a key output plot, showing how the estimated GP shape parameter varies as a function of the non-exceedance probability. This plot should be used to narrow down on an NEP range over which the GP shape is relatively stable.  In the left panel a reasonable choice might be [0.5,0.75], in the right panel a narrower range might be chosen, say [0.5, 0.65]. The right limit can usually be chosen as the last point before which the confidence interval widens or there is a change in slope. The lower limit should generally not be below the mode of the data since we are fitting a tail model. We choose to use an ensemble of thresholds in our analysis in recognition of the challenge of threshold selection in extreme value statistics.
	}%
}
\vspace{10pt}

\begin{figure}
%\begin{figure}
    \centering
    \begin{subfigure}{0.45\textwidth}
	\centering
	\includegraphics[width=\textwidth]{"./PPCGuideExamples/Tp_Hs_ByDrc/Figures/Stg3_Hs_7_ThresholdStability"}
	    \end{subfigure}
    \begin{subfigure}{0.45\textwidth}
		\includegraphics[width=\textwidth]{"./PPCGuideExamples/Tp_Hs_ByDrc/Figures/Stg3_Tp_7_ThresholdStability"}
    \end{subfigure}
	\caption{GP shape $\xi$ as a function of the NEP for 2 responses Hs (left plot) and Tp (right plot) from the North Sea data. Black dots show individual bootstrap estimates, red lines are local binned median, 2.5 and 97.5 percentile estimates. A well behaved model should be stable over a range of NEP's.}
	\label{fig:Stg3_Hs_6_ThresholdStability}
\end{figure}

Finally, Figure \ref{fig:Stg3_Hs_7_ReturnValueCDF} provides 10 and 100 year return level cumulative distribution functions for each covariate (here, directional) sector. When there are fewer colours in the plot than the legend; one or more of the CDFs overlap. Empty sectors are listed in the legend with an ``Empty Bin'' description and do not have an associated CDF curve. 

\begin{figure}
	\centering
	\includegraphics[width=0.9\textwidth]{"./PPCGuideExamples/Tp_Hs_ByDrc/Figures/Stg3_Hs_8_ReturnValueCDF"}
	\caption{Marginal 10 year (upper plot) and 100 year (lower plot) return value CDFs ($Hs$). Directional sectors are show using coloured lines. The black line shows the omni-directional estimate.}
	\label{fig:Stg3_Hs_7_ReturnValueCDF}
\end{figure}
\newpage
