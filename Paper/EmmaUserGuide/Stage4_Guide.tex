%-------------------------------------------------------------------------------
\subsection{Conditional extremes model}
%
The standard-scale (Gumbel or Laplace) sample $\{y_{1i},y_{2i} \ldots,y_{Di}\}_{i=1}^{N}$ above some threshold of the conditioning variate $Y_1$ is used to estimate a conditional extremes model with parameters $\boldsymbol{\alpha}_{\tilde{\tau}k},  \boldsymbol{\beta}_{\tilde{\tau}},  \boldsymbol{\mu}_{\tilde{\tau}}$ and $ \boldsymbol{\sigma}_{\tilde{\tau}}$
%
\pbe
(Y_2, Y_3 \dots Y_D| Y_1 = y_{1i}) = \boldsymbol{\alpha}_{\tilde{\tau} k} y_{1i} + y_{1i}^{\boldsymbol{\beta}_{\tilde{\tau}}} \boldsymbol{W}_{\tilde{\tau}}  \text{ for } y>\phi_{\tilde{\tau}  k}.
\pee
%
 Threshold $\phi_{\tilde{\tau} k}$ is defined as the quantile of the standard Gumbel distribution with non-exceedance probability $\tilde{\tau}$ for covariate bin $k$. Note that we have two distinct non-exceedance probabilities: one for the marginal model fitting ($\tau$) and a secondary one for the H\&T model ($\tilde{\tau}$) - the value these parameters take should not necessarily be the same. 
Parameter $\boldsymbol{\alpha} \in [0,1]$ captures the extent of positive linear dependence between the associated and conditioned variable (with a stronger positive relationship as $\alpha \rightarrow 1$). $\boldsymbol{\beta} \in (-\infty,1]$ captures the spread of data around that linear relationship, with large negative values indicating a very tight distribution of data particularly for higher values of the conditioning variable (\emph{extremal dependence}); and positive values indicating a large degree of variance around the relationship, again particularly for higher values of the conditioning variable (\emph{extremal independence}). An illustration of this varying nature of dependence which the H\&T model can capture is given in Figure \ref{fig:Stg4_HTDiagram}. 

Finally, $\boldsymbol{W}_{\tilde{\tau}}$ is a random variable with an unknown distribution, the density of which we estimate using residuals from the fitted model. For model-estimation only, we assume $\boldsymbol{W}_{\tilde{\tau}} \sim N(\boldsymbol{\mu}_{\tilde{\tau}}, \boldsymbol{\sigma}_{\tilde{\tau}}^2)$.

\begin{figure}
	\centering
	\includegraphics[width=\textwidth]{Figures/HTDiagram}
	\caption{Illustration of the impact of HT parameters $\alpha$ and $\beta$ on the structure of dependence between two standard-scale (in this case Gumbel) distributed random variables}
	\label{fig:Stg4_HTDiagram}
\end{figure}

Model-fitting therefore corresponds to estimating $\{\boldsymbol{\alpha},\boldsymbol{\beta}, \boldsymbol{\mu}, \boldsymbol{\sigma}\}$ given a sample of values for $\{Y_1, Y_2,\dots\,Y_{D}\}$. All of $\phi, \boldsymbol{\alpha}, \boldsymbol{\beta}, \boldsymbol{\mu}$ and $\boldsymbol{\sigma}$ are in principle functions of covariates. Using the conditional extremes model, we simulate joint extremes on the standard Gumbel or Laplace scale, and transform these realisations to the original scale using the probability integral transform once more. 

\subsection{Running Stage4}

Running Stage4 fits a Hefferenan and Tawn conditional extreme value model. Marginal model data and parameters (\verb|Output\MM1.mat| and \verb|Output\MM2.mat|) are loaded from stage 3.\\

\textbf{Run Scripts}: \verb|Stage4_FitHeffernanTawn.m|\\
\textbf{Output files}: \verb|Output\HT.mat|\\
\textbf{Inputs}: \\
\begin{itemize}	
	\item \verb|HTNEP| [scalar] conditional non exceedence probability range, make sure $>\exp(-\exp(-0))=0.368$ or the Gumbel transformation will fail;
	\item \verb|NonStationary| [boolean] If 0: use a stationary $\alpha$ parameter in H\&T model; if 1: fit penalised-piecewise-constant (non-stationary) $\alpha$ using the same bins as the marginal analysis;
	\item \verb|CV.CVMth| [boolean] If 0: Only cross validate roughness penalty for original dataset (fast); or 1: Cross Validate smoothness for every bootstrap re-sample (slow);
	 \item \verb|CV.nCV| [scalar] number of cross validation groups;
 	\item \verb|CV.nSmth| [scalar] number of roughnesses tried in CV;
 	\item \verb|CV.SmthLB| [scalar] lower bound (log10) for set of candidate smoothness penalties;
 	\item \verb|CV.SmthUB|  [scalar] upper bound (log10) or set of candidate smoothness penalties;
 	\item \verb|SampleLocalResid| [boolean] If this is set to true (or 1), when simulating under H\&T model residuals are resampled locally (from the current covariate bin); if false (or 0), residuals are sampled globally, i.e. from any bin. 
\end{itemize}

Note that sampling residuals locally (setting \verb|SampleLocalResid| = true) effectively gives a non-stationarity to the residual part of the H\&T model, thus improving the fit/simulation procedure. That said, in the presence of bins with very few observations, we advise that this functionality is turned off (set to false) as the simulated data for a bin with very few observations will come from resampling a very small set of residuals many times. In this case it is therefore better to sample globally to increase the number of residuals from which the simulation resamples. 

The number of bootstrap resamples is inherited from the marginal model settings. For this stage to run successfully, you must have used the same number of bootstraps in each marginal model run (already highlighted in Section{Sct:RunStg3}). 

\subsection{Outputs}
Running Stage 4 produces a MATLAB data file called \verb|HT.mat| with the following contents:
\begin{itemize}
	\item \verb|HT.Prm| [$nPrm \times (D-1) \times$ \verb|nBoot|] H\&T model parameters
	\item \verb|HT.Rsd| [$nB  \times  1$] cell array of residuals sampled for each bootstrap
	\item \verb|HT.Thr| [$nB  \times  D-1$] H\&T threshold used
	\item \verb|HT.NEP|  [$nB  \times  1$] non-exceedence probability range on (0,1)
	\item \verb|HT.nBoot| [scalar] number of bootstraps
	\item \verb|HT.X| [$N \times (D-1)  \times$ \verb|nBoot|] conditioned variable transformed to standard scale
	\item \verb|HT.Y| [$N \times (D-1)  \times$ \verb|nBoot|] associated variable transformed to standard scale
	\item \verb|HT.RV.X_Stn|: $[(B+1 )\times $ \verb|RV.nRls| $ \times R$] simulated return values for conditioned value on standard scale
	\item \verb|HT.RV.X|: [$(B +1 )\times $ \verb|RV.nRls| $ \times R$]   simulated return values for conditioned value on original scale	
	\item \verb|HT.RV.Y_Stn|: [$(B +1 )\times (D-1)\times$ \verb|RV.nRls| $ \times R$] simulated return values for associated variables (conditioned on main variable) on standard scale
	\item \verb|HT.RV.Y|: [$(B +1 )\times (D-1)\times $ \verb|RV.nRls| $ \times R$] simulated return values for associated variables (conditioned on main variable) on original scale
	\item \verb|HT.RV.nRls| [scalar] number of realisations used in return value simulation
	\item \verb|HT.n| [scalar] number of data observations $N$
	\item \verb|HT.nDmn| [scalar] number of variables (main and associated) $D$
	\item \verb|HT.SmpLclRsdOn| [boolean] flag for use of residual-sampling from local bin
	\item \verb|HT.nAlp|  [scalar] number of $\alpha$ parameters in the model
	\item \verb|HT.nPrm| [scalar] total number of H\&T model parameters
	\item \verb|HT.nBin|  [scalar] number of covariate bins
	\item \verb|HT.nRtr|  [scalar] number of return periods
	\item \verb|HT.NonStat| [boolean] non-stationary $\alpha$ parameter flag
	\item \verb|HT.A| [$N \times$ \verb|nBoot|] matrix containing the bin allocation of the data samples in each bootstrap
	\item \verb|HT.RsdInd| [\verb|nBoot| $\times 1$] cell containing indices of the bootstrap samples
	\item \verb|HT.CVMth| [boolean] cross-validation method used (see associated entry in \emph{Inputs} for more detail)
	\item \verb|HT.nCV| [scalar] number of cross-validation groups
	\item \verb|HT.nSmth| [scalar] number of smoothness parameters used
	\item \verb|HT.SmthLB| [scalar] lower bound for set of candidate smoothness penalties
	\item \verb|HT.SmthUB| [scalar] upper bound for set of candidate smoothness penalties
	\item \verb|HT.SmthSet| [$1\times$ \verb|nSmth|] set of candidate smoothness penalties
	\item \verb|HT.OptSmth| [$1 \times$ \verb|nBoot|] optimal smoothness penalty resulting from cross-validation
	\item \verb|HT.CVLackOfFit| [$nSmth\times$ \verb|nBoot|] lack of fit for roughness estimation
	\item \verb|HT.MarginType| [string] margin type for transformation to standard scale
\end{itemize}

\vspace{10pt}
	\noindent\fbox{%
		\parbox{\textwidth}{%
	Figure \ref{fig:Stg4_HT_1_SmlvsData} shows a comparison of the data and a simulation from the H\&T model. If the simulated red points do not reflect the distribution of the original data in black, consider reworking the inputs to improve the model fit. 
		}%
	}
\vspace{10pt}

\begin{figure}
	\centering
	\includegraphics[width=0.9\textwidth]{"./PPCGuideExamples/Tp_Hs_ByDrc/Figures/Stg4_HT_1_SmlvsData"}
	\caption{Comparison of original data (black) and simulation from fitted H\&T model for $10\times$ period of the data (red) on standard margins (upper plot) and original margins (bottom plot). On the original scale two different spikes can be seen at the upper right corner reflecting different marginal characteristics in T$_P$ }
	\label{fig:Stg4_HT_1_SmlvsData}
\end{figure}


On the original scale, two different spikes can be seen in the upper right hand tail of the joint distribution, reflecting different marginal characteristics in T$_P$.  

\vspace{10pt}
		\noindent\fbox{%
		\parbox{\textwidth}{%
	Plots of model parameter estimates and residual distributions as a function of threshold and covariate (direction) aid assessment of model-quality. Any inhomogeneity with respect to direction in Figure \ref{fig:Stg4_HT_2_ResidualDiagnostics2} suggests that the H\&T model has not fitted well. In this case, you should adjust the H\&T NEP range and possibly return to Stage 3 to adjust the covariate bin-edges to improve your representation of non-stationarity with respect to the covariate(s).
			}%
	}
\vspace{10pt}

\begin{figure}
	\centering
	\includegraphics[width=0.9\textwidth]{"./PPCGuideExamples/Tp_Hs_ByDrc/Figures/Stg4_HT_2_ResidualDiagnostics2"}
	\caption{Diagnostic of the residuals from the H\&T fitting. Left panel shows a histogram of the residuals. Middle panel shows a normal QQ plot.  The right panel shows residuals as a function of direction. It is typical that these residuals are quite skewed (not normal), which is why they are reused in the simulation procedure.}
	\label{fig:Stg4_HT_2_ResidualDiagnostics2}
\end{figure}

Figure \ref{fig:Stg4_HT_2_ResidualDiagnostics1} summarises the same residuals, but this time not breaking out by covariate. The residual distribution in the left subplot is compared against the normal distribution in the right subplot by way of a normal QQ plot. The residual distribution will typically not be well-matched to the line $y=x$ (i.e. not be normally distributed) as assumed when fitting the conditional model. This is why we reuse the residuals in the simulation procedure. 

\begin{figure}
	\centering
	\includegraphics[width=0.9\textwidth]{"./PPCGuideExamples/Tp_Hs_ByDrc/Figures/Stg4_HT_2_ResidualDiagnostics1"}
	\caption{Diagnostic of the residuals from the H\&T fitting: residuals plotted as a function of direction. It is typical that these residuals are quite skewed (not normal), which is why they are reused in the simulation procedure.}
	\label{fig:Stg4_HT_2_ResidualDiagnostics1}
\end{figure}



Figure \ref{fig:Stg4_HT_3_Parameters} shows model parameter $\boldsymbol{\alpha}$ for the stationary case is near 1, this indicates strong dependency between large H$_S$ and T$_P$. In the non-stationary case, Figure \ref{fig:Stg4_HT_3_ParametersNonStat} shows $\boldsymbol{\alpha}$ fairly similar in most sectors but it is highly uncertain in the sector where there is no data. $\boldsymbol{\alpha}$ nearer 0 would indicate weak or no dependency. The reader is directed back to Figure \ref{fig:Stg4_HTDiagram} for an illustration of the influence of parameters $\boldsymbol{\alpha}$ and $\boldsymbol{\beta}$ on the shape of dependence.

\begin{figure}
	\centering
	\includegraphics[width=0.9\textwidth]{"./PPCGuideExamples/Tp_Hs_ByDrc/Figures/Stg4_HT_3_Parameters_Tp"}
	\caption{Histograms of the H\&T parameters over bootstrap re-samples in a non-stationary case. The parameter uncertainty captures, marginal (bootstrap and NEP) and conditional (bootstrap and H\&T NEP) uncertainty.  }
	\label{fig:Stg4_HT_3_ParametersNonStat}
\end{figure}

\begin{figure}
	\centering
	\includegraphics[width=0.9\textwidth]{"./PPCGuideExamples/Tp_Hs_ByDrc/Figures/Stg4_HT_3_Parameters_Tp_Stationary"}
	\caption{Histograms of the H\&T parameters over bootstrap re-samples in a stationary case. The parameter uncertainty captures, marginal (bootstrap and NEP) and conditional (bootstrap and H\&T NEP) uncertainty.  }
	\label{fig:Stg4_HT_3_Parameters}
\end{figure}

\vspace{10pt}
\noindent\fbox{%
	\parbox{\textwidth}{%
	The threshold stability plots in Figures \ref{fig:Stg4_HT_4_AlphaThresholdStability} and \ref{fig:Stg4_HT_4_BetaThresholdStability} are similar to those in Figure \ref{fig:Stg3_Hs_6_ThresholdStability}. These should be used in the same way as described in section \ref{Sct:RunStg3} to find a suitable range for the H\&T NEP. A range of [0.5,0.7] would seem to be a reasonable choice here.
	}%
}
\vspace{10pt}

\begin{figure}
	\centering
	\includegraphics[width=0.9\textwidth]{"./PPCGuideExamples/Tp_Hs_ByDrc/Figures/Stg4_HT_4_AlphaThresholdStability_Tp"}
	\caption{H\&T parameter $\alpha$  as a function of the H\&T NEP. Black dots show individual bootstrap estimates, red lines are local binned median, 2.5 and 97.5 percentile estimates. A well behaved model should be stable over a range of NEP's }
	\label{fig:Stg4_HT_4_AlphaThresholdStability}
\end{figure}

\begin{figure}
	\centering
	\includegraphics[width=0.9\textwidth]{"./PPCGuideExamples/Tp_Hs_ByDrc/Figures/Stg4_HT_4_BetaThresholdStability"}
	\caption{H\&T parameter $\beta$ as a function of the H\&T NEP. Black dots show individual bootstrap estimates, red lines are local binned median, 2.5 and 97.5 percentile estimates. A well behaved model should be stable over a range of NEP's}
	\label{fig:Stg4_HT_4_BetaThresholdStability}
\end{figure}

 Figure \ref{fig:Stg4_HT_6_ConditionalReturnValueCDF} shows the return value CDFs for the North Sea T$_P$$|$H$_S$ example. Here the omni directional CDF is bimodal, this is largely due to directional differences in the T$_P$ marginal distribution. Similar effects can be seen in Figure \ref{fig:Stg4_HT_1_SmlvsData}.


\begin{figure}
	\centering
	\includegraphics[width=0.9\textwidth]{"./PPCGuideExamples/Tp_Hs_ByDrc/Figures/Stg4_HT_6_ConditionalReturnValueCDF"}
	\caption{Conditional return value CDF's $p(\textrm{Tp} | Hs_{10})$ and $p(\textrm{Tp} | Hs_{100})$. Directional CDFs are shown using coloured lines. Black line shows the omni-directional estimate. In this case using the North Sea data the omni directional CDF is bimodal, this is largely due to marginal differences in Tp. Similar effects can be seen in figure \ref{fig:Stg4_HT_1_SmlvsData}}
	\label{fig:Stg4_HT_6_ConditionalReturnValueCDF}
\end{figure}

\newpage